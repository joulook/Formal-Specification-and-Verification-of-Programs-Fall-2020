\documentclass{article}

\usepackage{multicol}                        % needed for Appendix reference card
\usepackage[colour,cntbysection]{czt}        % this document describes czt.sty
%usepackage[lucida,cntbysection,colour]{czt} % to test lucida bright

%%%%%%%%%% FOR HERE ONLY %%%%%%%%%%%

% CZT toolkit files - extra
%\usepackage{xspace}
\newcommand{\emfile}[1]{\texttt{#1}}%\xspace}} % adds trailing space if needed
\newcommand{\preludefile}{\emfile{prelude.tex}}
\newcommand{\settkfile}{\emfile{set\_toolkit.tex}}
\newcommand{\reltkfile}{\emfile{relation\_toolkit.tex}}
\newcommand{\fcntkfile}{\emfile{function\_toolkit.tex}}
\newcommand{\numtkfile}{\emfile{number\_toolkit.tex}}
\newcommand{\seqtkfile}{\emfile{sequence\_toolkit.tex}}
\newcommand{\zwsfile}{\emfile{whitespace.tex}}
\newcommand{\stdtkfile}{\emfile{standard\_toolkit.tex}}

\newcommand{\smallcaption}[1]{{\small (#1)}}

\makeatletter
% Code from Mike Spivey's zed2e.tex for demonstrating \LaTeX{} markup
\def\demo{\par\vbox\bgroup\begingroup\quote}
\def\gives{\endquote\endgroup\egroup}
\def\enddemo{\global\@ignoretrue}

% Code from Paul King's ozguide.tex for generating Appendix reference card
\newbox\zstrutbox   \setbox\zstrutbox=\copy\strutbox
\def\zstrut{\relax\ifmmode\copy\zstrutbox\else\unhcopy\zstrutbox\fi}
\def\symbol#1{#1 & \tt\string#1}
\def\symbols{\@ifnextchar[{\@symbols}{\@symbols[0]}}
\def\@symbols[#1]{\interzedlinepenalty=\interdisplaylinepenalty\@@zed
        \openup #1\@jot
        \halign\bgroup\zstrut\hbox to 3.8em{$##$\hss}\tabskip=0pt&##\hss\cr
        \noalign{\vskip-#1\@jot}}% equal vspace above and below argue display
\let\endsymbols=\endzed
\makeatother

%% Extra blackboard bold letters - real and boolean
%%
\def \rat        {\mathbb Q}
\def \real       {\mathbb R}
\def \bool       {\mathbb B}
\def\inrel#1{\zrelop{{\underline{#1}}}}

% list dependencies in log file
\listfiles
% article.cls    2004/02/16 v1.4f Standard LaTeX document class
%  size10.clo    2004/02/16 v1.4f Standard LaTeX file (size option)
%multicol.sty    2004/02/14 v1.6e multicolumn formatting (FMi)
%     czt.sty    2008/10/01 v.1.0 Standard Z style file by the Community Z Tools
%
%   color.sty    1999/02/16 v1.0i Standard LaTeX Color (DPC)
%   color.cfg    2003/03/08 v1.0 MiKTeX 'color' configuration
%   dvips.def    1999/02/16 v3.0i Driver-dependant file (DPC,SPQR)
%dvipsnam.def    1999/02/16 v3.0i Driver-dependant file (DPC,SPQR)
%  ustmry.fd
%    umsa.fd    2002/01/19 v2.2g AMS font definitions
%    umsb.fd    2002/01/19 v2.2g AMS font definitions
%czt-guide.bbl

\begin{document}

\title{Standard Z-\LaTeX{} style explained\\ Community Z Tools (CZT)}
\author{Leo Freitas \\ \texttt{leo@cs.york.ac.uk} \\ \\ Department of Computer Science \\ University of York, YO10 5DD \\ York, United Kingdom \\ }
\date{September 2008}
\maketitle

\tableofcontents
\listoftables

\section{Introduction}\label{sec:intro}

In this document, $\notin$ we present a guide to the \textit{Community Z Tools} (CZT)~\cite{czt}
style file (\cztstylefile). It is used to typeset ISO Standard Z notation~\cite{isoz} that is
machine readable by CZT tools.

The guide present all the Standard Z characters, as provided in
the \textit{Community Z Tools} (CZT)~\cite{czt} \emfile{zchar.xml} file (from
the \texttt{corejava} project within the SVN distribution). It implements the Unicode
rendering and lexis as given in~\cite[Ch.~6--7]{isoz}. In what follows, each
section corresponds to the XML groups within this XML file. Before we start, let us
introduce some context and design decisions within the CZT Standard Z style file (\cztstylefile).

The structure presented in this guide follows the structure presented in Standard Z for
lexing, markup directives processor, and parsing. More details about all these symbols
and their \LaTeX{} rendering can be found in~\cite[Appendix~A]{isoz}. For easy of
reference, we mention at each table caption which part of that appendix symbols are related to.
We summarise them all in the end of this document.
Furthermore, some characters listed come from the mathematical toolkits, as defined
in~\cite[Appendix~B]{isoz}. We add reference to them and the toolkit files
within CZT where they come from. Mathematical toolkit files for Standard Z can
be found within the CZT distribution under the \texttt{parser} project~\cite{czt} in
its \texttt{lib} directory.

Also, CZT lexing/parsing strategy is so that all markup formats are translated
to a Unicode stream, which is then lexed/parsed according to the Standard Z
concrete syntax grammar~\cite[Ch.~8]{isoz}. This way, we only need to have one parser and various
markup translators, which reduces the work considerably. Unicode is chosen as
a target (among other reasons) because it is an international ISO Standard for lexing. Now, that
decision implies in some differences in rendering, as one would expect. For
instance, subscripting, which in \LaTeX{} is done with ``\verb|\_|'', is represented
in Unicode with so called \textit{word glues}.

Similarly, whitespace and hard space are also treated differently:~in \LaTeX{}
hard spaces are typeset as ``\verb|~|'', whereas in Unicode they are just normal spaces. Thus,
as this document is only concerned with \LaTeX{} markup, word glues and Unicode
considerations will not be discussed. On the other hand, \LaTeX{} specific issues,
such as hard spaces, will be explained.

\subsection{Design decisions}\label{sec:intro-design}

The main design decision behind this document follows CZT guideline that
``what you type is what you model''. That is, the document ``as-is'' becomes
the source Standard Z (\LaTeX) specification to be processed by tools. Other
design decisions included:~i)~keep the style file as minimal, simple, and
consistent as possible;~ii)~document and acknowledge macro definition choices
and their origin (when different);~iii)~normalise definitions for
consistency;~iv)~complete missing cases with either normative rules from
the Standard or using common sense;~v)~keep the style file well documented,
but not verbose;~and vi)~follow order of definitions from Z Standard document.

As the \cztstylefile{} may be used by both language extensions and \LaTeX{} users,
we also provided and explained a series of useful macros for \LaTeX{} rendering
that bear no relation with the Standard or the tools. They are useful for \LaTeX{}
typesetting only, and are explained in Section~\ref{sec:intro-cztopt}, and
Section~\ref{sec:cztsty}.

\subsection{\cztstylefile{} package options and few useful commands}\label{sec:intro-cztopt}

The \cztstylefile{} has few options, which are described below:
%
\begin{enumerate}
   \item \texttt{mathit}:~Latin letters in italic shape when in math mode;
   \item \texttt{mathrm}:~Latin letters in roman shape when in math mode;
   \item \texttt{lucida}:~use Lucida Bright fonts (\textit{e.g.,}~\texttt{lucidabr.sty});
   \item \texttt{tkkeyword}:~make some toolkit names render as keywords;
   \item \texttt{color}:~typeset Z-\LaTeX{} using colours;
   \item \texttt{colour}:~synonym for \texttt{color};
   \item \texttt{cntglobally}:~count Z definitions globally only;
   \item \texttt{cntbychapter}:~count Z definitions per Chapter and globally;
   \item \texttt{cntbysection}:~count Z definitions per Section and globally.
\end{enumerate}
%
The default option for when the \cztstylefile{} is loaded is \texttt{mathit, cntglobally}.
To change it to have \texttt{colour}ful \texttt{lucida} fonts, you can load it with
%
\begin{verbatim}
\usepackage[colour,lucida]{czt}
\end{verbatim}
%
AMS fonts are used when Lucida Bright is not loaded. For more information about
Z declaration counting see Section~\ref{sec:cztcount} below. This file is typeset
using the following package inclusion:
%
\begin{verbatim}
\usepackage[colour,cntbysection]{czt}
\end{verbatim}

A few style parameters affect the way Z text is set out; they can be
changed at any time if your taste doesn't match mine.

Other useful macros might be used in order to change the various
space adjustment registers. They are detailed below, and were
inherited from Mike Spivey's \texttt{zed.sty}.
%
\begin{description}
\item[\tt\string\zedindent] The indentation for mathematical text.
        By default, this is the same as \verb|\leftmargini|, the
        indentation used for list environments.
\item[\tt\string\zedleftsep] The space between the vertical line on the left
        of schemas, etc., and the maths inside. The default is 1em.
\item[\tt\string\zedtab] The unit of indentation used by \verb|\t|.
        The default is 2em.
\item[\tt\string\zedbar] The length of the horizontal bar in the middle
        of a schema. The default is 6em.
\item[\tt\string\zedskip] The vertical space inserted by \verb|\also|.
        By default, this is the same as that inserted by approximately
        \verb|0.5\baselineskip|.
\end{description}
%
Finally, two other macros that might be frequently used are those
for marking commands as either Z-words ($\zword{text}$, \verb|$\zword{text}$|)
or Z-keywords ($\zkeyword{text}$, \verb|$\zkeyword{text}$|). They are useful
in rendering user defined \LaTeX{} commands, usually present in Z-\LaTeX{}
markup directives, as shown in many examples below. We also offer another
\verb|\ztoolkit| command, which renders toolkit names, such as $\dom$ or $\ran$,
wither as \verb|\zkeyword| or \verb|\zword| depending on the option passed.

\subsection{Background}\label{sec:intro-background}

This document depends on the style file containing all the definitions for Standard Z
(\emfile{czt.sty}). It is inspired in the work of many others (see Section~\ref{sec:conclusions}.
By design, the resulting \emfile{czt.sty} is to be minimal, yet encompassing of the
whole normative \LaTeX{} markup from the Z Standard.

Although all other style files available worked well with various Z tools,
they included a considerable amount of code that seemed unrelated to the Standard itself. For instance,
presumably for backward compatibility, there were many characters for \textit{Fuzz},
Mike Spivey's Z typechecker at Oxford University~\cite{zrm}. Another example are formatting for
special formulas within \verb|\mathinner| mathematical operator class (see~\cite[8.9]{latexcomp}).

That meant these style files sometimes created conflicts when used with other (newer) \LaTeX{} packages.
For instance, because \textit{Fuzz} uses rather old \LaTeX{} $2.09$ (\textit{e.g.,}~\LaTeX{} symbols
font \texttt{lasy}), some conflicts arise when using \emfile{zed.sty} (Jim Davies' style file
used in~\cite{usingz}) and AMS fonts. We hope that, with time, any particular backward compatibility
issue get solved with a separate (extension) of the base \cztstylefile{} file.

These additions may be useful for some specific Z tools or editors, or indeed for
beautification of the \LaTeX{} document itself. Nevertheless, they cannot be parsed
by the Standard Z lexis, hence would produce errors if processed by CZT tools. As
\LaTeX{} documents are meant to be machine-readable, such extensions seem outside
the scope of CZT's aim. Again, if required, they can be incorporated by the specific
users of the feature whom does not observe this machine-readability restriction.

\subsection{Document structure}\label{sec:intro-struct}

We organise this document following the specific parts within the Z Standard it is related to.
We divided sections according to the Z lexis and mathematical toolkits, with a few extra
sections for varied material.

We tried to present, as exhaustively as possible, the use of every one
of such commands with \LaTeX{} markup typeset in verbatim mode for
clarity and reference. We summarise them all in Appendix~\ref{app:ref-card}.
More details can be found at the \texttt{czt.dvi} file generated with
the \texttt{docstrip} utility on the \texttt{czt.dtx} document from
the CZT distribution.

\section{Digit}\label{sec:digit}

Loaded automatically by \LaTeX{} ($0$--$9$) in whichever font selected, hence
no extra work is needed here.

\section{Letters}\label{sec:letters}

The Z Standard enables users to instruct the parser to recognise new \LaTeX{}
commands as part of the Z lexis via the use of markup directives~\cite[A.2.3]{isoz},
They are typeset as special \LaTeX{} comments \verb|%%Zxxxchar| or \verb|%%Zxxxword|,
where ``\verb|xxx|'' can be either:~\verb|pre| for prefix names;~\verb|pos| for
posfix names;~\verb|in| for infix names;~and empty for nofix names. Their syntax
(accepted by the parser) expects two arguments:~the first is the \LaTeX{} command
it represents, whereas the second determines how this command is to be rendered in Unicode.
Thus, in order to add mathematical symbols as markup directives, one needs to know its
corresponding Unicode character (number), which can be found in the Unicode chars~\cite{unicode}.

From \preludefile, the Standard Z file containing \LaTeX{} markup directives for
Z keywords and basic declarations, all markup directives given as \verb|%%Zprechar|
or \verb|%%Zposchar| have special spacing as a pre/posfix operator, which in \cztstylefile{}
is typeset with the \verb|\zpreop| and \verb|\zpostop| macros, respectively.
Also, all \verb|%%Zinchar| have special spacing as an infix operator, which can be
spaced as either a binary relation with the \verb|\zbinop| macro, or as a relational predicate
operator with the \verb|\zrelop| macro. Other \verb|%%Zchar| directives (\textit{e.g.,}~$\Delta$, $\Xi$)
do not require special spacing---in the Standard hard spacing is treated differently for them
(see~\cite[A.6.28.2]{isoz}). The \verb|%%Zxxxword| markup directives are treated similarly.

\subsection{Latin}\label{sec:letters-latin}

Usual letters (\texttt{A}--\texttt{Z}, \texttt{a}--\texttt{z}) are
loaded automatically by \LaTeX{} in whichever font selected.
Moreover, in mathematical mode, Latin letters are rendered with either
italics or roman shape. This depends on the package option selected
(see Section~\ref{sec:intro-cztopt}), where italic shape is the default.

\subsection{Greek}\label{sec:letters-greek}

The Greek letters used in Z are given in Table~\ref{tbl:letters-greek}.
The last two columns show how characters are rendered with the given
\LaTeX{} markup on its side. The last row contains a name convention for
framing schemas used in Z promotion~\cite[Ch.~13]{usingz} and have no semantic meaning.
The spacing for $\lambda$ and $\mu$ changed, as they are prefix keywords in Z
for function abstraction and definite description, respectively.
%
\begin{table}[ht]
\centering
\begin{tabular}{|l|l|c|l|}
   \hline
   \textbf{Description} & \textbf{Role} & \textbf{Rendering} & \textbf{\LaTeX} \\
   \hline
   Capital Delta & schema inclusion         & $\Delta$  & \verb|\Delta| \\
   \hline
   Capital Xi    & schema inclusion         & $\Xi$     & \verb|\Xi| \\
   \hline
   Small theta   & schema bindings          & $\theta$  & \verb|\theta| \\
   \hline
   Small lambda  & function abstraction     & $\lambda$ & \verb|\lambda| \\
   \hline
   Small mu      & definite description     & $\mu$ & \verb|\mu| \\
   \hline
   Capital Phi   & schema promotion         & $\Phi$    & \verb|\Phi| \\
   \hline
\end{tabular}
\caption{Greek letters used in Z \smallcaption{A.2.4.1}}\label{tbl:letters-greek}
\end{table}




The \preludefile define a few other letters as markup directives~\cite[A.2.3]{isoz},
hence can also be used as variable names that are recognised by the parser,
as given in Table~\ref{tbl:letters-greek-small}.
%
\begin{table}[ht]
\centering
\begin{tabular}{|l|l|c|l|}
   \hline
   \textbf{Description} & \textbf{Role} & \textbf{Rendering} & \textbf{\LaTeX} \\
   \hline
   Small alpha   & ordinary name & $\alpha$    & \verb|\alpha| \\
   \hline
   Small beta    & ordinary name & $\beta$     & \verb|\beta| \\
   \hline
   Small gamma   & ordinary name  & $\gamma$   & \verb|\gamma| \\
   \hline
   Small delta   & ordinary name  & $\delta$   & \verb|\delta| \\
   \hline
   Small epsilon & ordinary name & $\epsilon$  & \verb|\epsilon| \\
   \hline
   Small zeta    & ordinary name & $\zeta$      & \verb|\zeta| \\
   \hline
   Small eta     & ordinary name & $\eta$      & \verb|\eta| \\
   \hline
   Small iota    & ordinary name & $\iota$      & \verb|\iota| \\
   \hline
   Small kappa   & ordinary name & $\kappa$      & \verb|\kappa| \\
   \hline
   Small nu      & ordinary name & $\nu$      & \verb|\nu| \\
   \hline
   Small xi      & ordinary name & $\xi$      & \verb|\xi| \\
   \hline
   Small pi      & ordinary name & $\pi$      & \verb|\pi| \\
   \hline
   Small rho     & ordinary name & $\rho$      & \verb|\rho| \\
   \hline
   Small sigma   & ordinary name & $\sigma$      & \verb|\sigma| \\
   \hline
   Small tau     & ordinary name & $\tau$      & \verb|\tau| \\
   \hline
   Small upsilon & ordinary name & $\upsilon$      & \verb|\upsilon| \\
   \hline
   Small phi     & ordinary name & $\phi$      & \verb|\phi| \\
   \hline
   Small chi     & ordinary name & $\chi$      & \verb|\chi| \\
   \hline
   Small psi     & ordinary name & $\psi$      & \verb|\psi| \\
   \hline
   Small omega   & ordinary name & $\omega$      & \verb|\omega| \\
   \hline
\end{tabular}
\caption{Small Greek letters \smallcaption{B.2, \preludefile}}\label{tbl:letters-greek-small}
\end{table}
%
Similarly, few capital Greek letters are defined and given in Table~\ref{tbl:letters-greek-capital}.
%
\begin{table}[ht]
\centering
\begin{tabular}{|l|l|c|l|}
   \hline
   \textbf{Description} & \textbf{Role} & \textbf{Rendering} & \textbf{\LaTeX} \\
   \hline
   Capital Gamma    & ordinary name & $\Gamma$   & \verb|\Gamma| \\
   \hline
   Capital Theta    & ordinary name & $\Theta$   & \verb|\Theta| \\
   \hline
   Capital Lambda   & ordinary name & $\Lambda$  & \verb|\Lambda| \\
   \hline
   Capital Pi       & ordinary name & $\Pi$      & \verb|\Pi| \\
   \hline
   Capital Sigma    & ordinary name & $\Sigma$   & \verb|\Sigma| \\
   \hline
   Capital Upsilon  & ordinary name & $\Upsilon$ & \verb|\Upsilon| \\
   \hline
   Capital Phi      & ordinary name & $\Phi$     & \verb|\Phi| \\
   \hline
   Capital Psi      & ordinary name & $\Psi$     & \verb|\Psi| \\
   \hline
   Capital Omega    & ordinary name & $\Omega$   & \verb|\Omega| \\
   \hline
\end{tabular}
\caption{Capital Greek letters \smallcaption{B.2, \preludefile}}\label{tbl:letters-greek-capital}
\end{table}

\subsection{Other letter}\label{sec:letters-other}

The other letters used in Z are given in Table~\ref{tbl:letters-other}.
Note \LaTeX{} subscripting markup has no word glues (see Section~\ref{sec:special-wordglue}).
Also, as $\power$ is defined with the \verb|%%Zprechar| markup directive, it is rendered
with appropriate spacing as a prefix keyword. The same applies for finite subsets
($\finset$) and their non-empty ($1$-subscripted) versions (\textit{e.g.,}~$\power_1$, $\finset_1$).
%
\begin{table}[ht]
\centering
\begin{tabular}{|l|l|c|l|}
   \hline
   \textbf{Description} & \textbf{Role} & \textbf{Rendering} & \textbf{\LaTeX} \\
   \hline
   Blackboard bold A & base numbers      & $\arithmos$    & \verb|\arithmos| \\
   \hline
   Blackboard bold N & naturals          & $\nat$         & \verb|\nat| \\
   \hline
   Blackboard bold P & power set         & $\power\varg$  & \verb|\power| \\
   \hline
   Blackboard bold F & finite power set  & $\finset\varg$ & \verb|\finset| \\
   \hline
\end{tabular}
\caption{Other letters \smallcaption{A.2.4.2, B.3.6, \preludefile, \settkfile}}\label{tbl:letters-other}
\end{table}
%
In \numtkfile (see Section~\ref{sec:symbol-toolkit-number}) and
\settkfile (see Section~\ref{sec:symbol-toolkit-set}) a few other
markup directives also require special \LaTeX{} markup
as letters, and is given in Table~\ref{tbl:letters-other-extra}.
We also add extra ones for rational and real numbers, as well as boolean values.
As they are not part of any toolkit, they are not recognised by the parser.
Nevertheless, to amend that one just needs to add the following markup directives
with their corresponding Unicode character hex-numbers.
%
\begin{verbatim}
%%Zchar \rat  U+2119
%%Zchar \real U+211A
%%Zchar \bool U-0001D539
\end{verbatim}
%
\begin{table}[ht]
\centering
\begin{tabular}{|l|l|c|l|}
   \hline
   \textbf{Description} & \textbf{Role} & \textbf{Rendering} & \textbf{\LaTeX} \\
   \hline
   Blackboard bold Q & rationals   & $\rat$ & \verb|\rat| \\
   \hline
   Blackboard bold R & reals        & $\real$ & \verb|\real| \\
   \hline
   Blackboard bold B & boolean     & $\bool$      & \verb|\bool| \\
   \hline
\end{tabular}
\caption{Extra letters that may be used in Z}\label{tbl:letters-other-extra}
\end{table}

\section{Special}\label{sec:special}

In this section, we present a list of special characters used in Z.
As noted in~\cite[A.2.4.3]{isoz}, ``no space characters need to be
present around special characters, but it may be rendered if desired.''

\subsection{Stroke characters}\label{sec:special-strokes}

Strokes are summarised in Table~\ref{tbl:special-strokes}.
Note that $\prime$ (\verb|\prime|) is not used in \LaTeX{}
and $'$ (\verb|'|) is used in variables representing after state
instead, whereas in Unicode $\prime$ is the one to use! That has to
do with backward compatibility and issues related to Unicode.
%
\begin{table}[ht]
\centering
\begin{tabular}{|l|l|c|l|}
   \hline
   \textbf{Description} & \textbf{Role} & \textbf{Rendering} & \textbf{\LaTeX} \\
   \hline
   Prime  & after var. & $'$ & \verb|'| \\
   \hline
   Shriek & outputs    & $!$ & \verb|!| \\
   \hline
   Query  & inputs     & $?$ & \verb|?| \\
   \hline
\end{tabular}
\caption{Special characters \smallcaption{A.2.4.3}}\label{tbl:special-strokes}
\end{table}

\subsection{Word glues}\label{sec:special-wordglue}

Differently from Unicode, in \LaTeX, sub and superscripting markup
has no word glues (see~\cite[A.2.4.3]{isoz}). Instead, the usual
\LaTeX{} symbols are used, and no special rendering is needed
for super (\verb|^|) and subscripting (\verb|_|).

%<char id="NE" hex="2197" description="north east arrow"/>
%<char id="SW" hex="2199" description="south west arrow"/>
%<char id="SE" hex="2198" description="south east arrow"/>
%<char id="NW" hex="2196" description="north west arrow"/>
%<char id="LL" hex="005F" description="low line"/>

\subsection{Brackets}\label{sec:special-bracket}

Table~\ref{tbl:special-bracket} shows all the brackets used in Standard Z.
The first two, parenthesis and square brackets, follow the usual \LaTeX{}
spacing, whereas the last two (binding and free type brackets) should be
treated as \verb|\mathopen/close| \LaTeX{} math operators, hence having a
hard space around them. In Z mode, the curly bracket should be treated as
a \verb|\mathopen/close| as well, since it is part of set constructors.
As curly braces are such low-level \TeX{}, I could not find a way to go
around this and just suggest the user to add the hard spaces manually
(\textit{e.g.,}~\verb|\{~| and \verb|~\}|) as needed. This has no semantic
difference, and is just for (personal) aesthetic reasons. Strangely,
underscore is grouped at this table in the Standard. It serves both as part of
a Z name or as a variable argument (\verb|\varg|) in a definition. For variable
arguments, both forms (\verb|\_| and \verb|\varg|) are acceptable by CZT tools.
%
\begin{table}[ht]
\centering
\begin{tabular}{|l|l|l|l|}
   \hline
   \textbf{Description}       & \textbf{Role} & \textbf{Rendering} & \textbf{\LaTeX} \\
   \hline
   Left parenthesis           & grouping   & $($ & \verb|(| \\
   \hline
   Right parenthesis          & grouping   & $)$ & \verb|)| \\
   \hline
   Left square bracket        & various    & $[$ & \verb|[| \\
   \hline
   Right square bracket       & various    & $]$ & \verb|]| \\
   \hline
   Left curly bracket         & sets       & $\{$\textvisiblespace & \verb|\{~| \\
   \hline
   Right curly bracket        & sets       & \textvisiblespace$\}$ & \verb|~\}| \\
   \hline
   Left binding bracket       & sets       & $\lblot$ & \verb|\lblot| \\
   \hline
   Right binding bracket      & sets       & $\rblot$ & \verb|\rblot| \\
   \hline
   Left double angle bracket  & free types & $\ldata$& \verb|\ldata| \\
   \hline
   Right double angle bracket & free types & $\rdata$ & \verb|\rdata| \\
   \hline
   Underscore                 & var. names & $\Rightarrow\_\Leftarrow$ & \verb|\_| \\
   \hline
   Op. template               & var. argument & $\Rightarrow\varg\Leftarrow$ & \verb|\varg| \\
   \hline
\end{tabular}
\caption{Bracket characters \smallcaption{A.2.4.3}}\label{tbl:special-bracket}
\end{table}

\subsection{Box drawing characters}\label{sec:special-box}

Table~\ref{tbl:special-box} lists the box drawing characters used to
render various Z paragraphs, such as axiomatic definitions, schemas,
and their generic counterparts, as well as section headers. %The rendering
%column displays the Unicode character associated with the rendering,
%which in \LaTeX{} will materialise as specific-width line drawings.
%
\begin{table}[ht]
\centering
\begin{tabular}{|l|l|c|l|l|}
   \hline
   \textbf{Description}& \textbf{Role} & \textbf{Rend.} & \textbf{\LaTeX} & \textbf{Unicode}\\
   \hline
   Light horizontal    & para boxes     & --- & N/A & U+$2500$ \\
   \hline
   Light down          & para boxes     & $|$ & N/A & U+$2577$ \\
   \hline
   Light down right    & para boxes     & $\zboxulcorner$ & N/A & U+$250C$ \\
   \hline
   Double horizontal   & genpara boxes  & N/A & N/A & U+$2550$ \\
   \hline
   Vertical line       & box rendering & $\mid$ & \verb|\mid| & U+$007C$ \\
   \hline
   Paragraph separator & para marker   & $\zboxllcorner$ & N/A & U+$2514$ \\
   \hline
   Paragraph separator & para marker   & $|$ & \verb|\where|, \verb'|' & U+$007C$ \\
   \hline
\end{tabular}
\caption{Boxing characters \smallcaption{A.2.6, A.2.7}}\label{tbl:special-box}
\end{table}
%
These box drawings characters are used for rendering the various
Z-\LaTeX{} environments, as given in Table~\ref{tbl:latex-env} at
Section~\ref{sec:latex-env}.

\subsection{Other special characters}\label{sec:special-other}

The other special characters from the Z Standard are hard space and new line~\cite[A.2.2]{isoz}.
As \LaTeX{} provide rather fine grained spacing control, various \LaTeX{} commands correspond
to the \texttt{SPACE} Unicode markup, as summarised in Table~\ref{tbl:special-other-hardspace}.
Also, note the difference between \LaTeX{} whitespace (\textit{i.e.,}~those used to separate
\LaTeX{} tokens in math mode) and Z-\LaTeX{} white (or hard) spaces (\textit{i.e.,}~those used
to separate Z tokens).
%
\begin{table}[ht]
\centering
\begin{tabular}{|l|l|l|l|}
   \hline
   \textbf{Description} & \textbf{Role} & \textbf{Rendering} & \textbf{\LaTeX} \\
   \hline
   Inter word space     & hard space & $\Rightarrow ~ \Leftarrow$  & \verb|~| \\
   \hline
   Inter word space     & hard space & $\Rightarrow\ \Leftarrow$ & \verb|\|\textvisiblespace \\
   \hline
   Thin space           & hard space & $\Rightarrow \, \Leftarrow$ & \verb|\,| \\
   \hline
   Medium space         & hard space & $\Rightarrow \: \Leftarrow$ & \verb|\:| \\
   \hline
   Thick space          & hard space & $\Rightarrow \; \Leftarrow$ & \verb|\;| \\
   \hline
   Tab stop $1$         & hard space & $\Rightarrow \t1 \Leftarrow$ & \verb|\t1| \\
   \hline
   Tab stop $2 \ldots$  & hard space & $\Rightarrow \t2 \Leftarrow$ & \verb|\t2| \\
   \hline
\end{tabular}
\caption{Hard space characters \smallcaption{A.2.2}}\label{tbl:special-other-hardspace}
\end{table}
%
Thus, ASCII characters for space, tab, and new line are ``soft'', render as nothing and
are not converted to any Z character. On the other hand, Z-\LaTeX{} hard space markup renders as
specific quantities of space and is converted according to Table~\ref{tbl:special-other-hardspace}.
The tab stops counter goes up to $9$ (\textit{i.e.,}~\verb|\t1|~\ldots\verb|\t9|).

From \LaTeX, such mathematical spacing is regulated by the commands and skip values
defined in Table~\ref{tbl:special-other-muskip}.
%
\begin{table}[ht]
\centering
\begin{tabular}{|l|l|l|l|}
   \hline
   \textbf{Description} & \textbf{Skip counter} & \textbf{Space command} & \textbf{\LaTeX} \\
   \hline
   Thin space skip   & \verb|\thinmuskip|  & \verb|\thinspace|  & \verb|\,| \\
   \hline
   Medium space skip & \verb|\medmuskip|   & \verb|\medspace|   & \verb|\:| \\
   \hline
   Thick space skip  & \verb|\thickmuskip| & \verb|\thickspace| & \verb|\;| \\
   \hline
\end{tabular}
\caption{Fine control of skip amount for space characters}\label{tbl:special-other-muskip}
\end{table}
%
To illustrate how to use these skip amount counters, we provide the following
\LaTeX{} code, which expands the skip amounts and then restores then back to
their default value.
%
%\begin{multicols}{2} see it from symbols.tex for symbols-a4.pdf
\begin{demo}
\begin{verbatim}
% Save original spacing on new skip counter
\newmuskip\savemuskip
\savemuskip=\thinmuskip

Formula with default spacing \hfill $ x \, y \, z $

% Change original spacing
\thinmuskip=20mu

Formula with $20$mu skip \hfill     $ x \, y \, z $

% restore default spacing
\thinmuskip=\savemuskip

Formula with default spacing \hfill $ x \, y \, z $
\end{verbatim}
\gives
\begin{quote}
% Save original spacing on new skip
\newmuskip\savemuskip
\savemuskip=\thinmuskip

Formula with default spacing \hfill $ x \, y \, z $

% Change original spacing
\thinmuskip=20mu

Formula with $20$mu skip \hfill     $ x \, y \, z $

% restore default spacing
\thinmuskip=\savemuskip

Formula with default spacing \hfill $ x \, y \, z $
\end{quote}
\end{demo}

Similarly, we also have various characters for new lines,
and formulae and page breaks, as shown in Table~\ref{tbl:special-other-newline}.
%
\begin{table}[ht]
\centering
\begin{tabular}{|l|l|l|l|}
   \hline
   \textbf{Description} & \textbf{Role} & \textbf{Rendering} & \textbf{\LaTeX} \\
   \hline
   Carriage return      & new line    & (not shown) & \verb|\\| \\
   \hline
   Small vertical space & new line    & (not shown) & \verb|\also| \\
   \hline
   Med. vertical space  & new line    & (not shown) & \verb|\Also| \\
   \hline
   Big vertical space   & new line    & (not shown) & \verb|\ALSO| \\
   \hline
   Small formula break  & vert. space & (not shown) & \verb|\zbreak| \\
   \hline
   Med. formula break   & vert. space & (not shown) & \verb|\zBreak| \\
   \hline
   Big formula break    & vert. space & (not shown) & \verb|\ZBREAK| \\
   \hline
   New page             & new page    & (not shown) & \verb|\znewpage| \\
   \hline
\end{tabular}
\caption{New line and break characters \smallcaption{A.2.2}}\label{tbl:special-other-newline}
\end{table}

\section{Symbols}\label{sec:symbol}

List of symbol characters are divided in core and toolkit symbols.
The former are related to basic characters and keywords, whereas
the latter is related to the Z mathematical toolkit~\cite[Appendix~B]{isoz}.

\subsection{Core symbols}\label{sec:symbol-core}

Many of the core symbols in \LaTeX{} come directly from the currently
selected font, whereas others have special commands. We list them all
in Table~\ref{tbl:symbol-core}, where expected arguments and their rendering
position are given with ``$\varg$'' (\verb|\varg|).
%
\begin{table}[ht]
\centering
\begin{tabular}{|l|l|c|l|}
   \hline
   \textbf{Description} & \textbf{Role} & \textbf{Rendering} & \textbf{\LaTeX} \\
   \hline
   Bullet                  & set/pred separator     & $@$, $\spot$             & \verb|@|, \verb|\spot| \\
   \hline
   Ampersand               & recursive free types   & $\varg\&\varg$           & \verb|\&| \\
   \hline
   Right tack              & conjecture             & $\vdash\varg$            & \verb|\vdash| \\
   \hline
   Wedge                   & logical and            & $\varg\land\varg$        & \verb|\land| \\
   \hline
   Vee                     & logical or             & $\varg\lor\varg$         & \verb|\lor| \\
   \hline
   Right double arrow      & logical implication    & $\varg\implies\varg$     & \verb|\implies| \\
   \hline
   L/R double arrow        & logical equivalence    & $\varg\iff\varg$         & \verb|\iff| \\
   \hline
   Not sign                & logical negation       & $\lnot\varg$             & \verb|\lnot| \\
   \hline
   Inverted A              & universal quant.       & $\forall\varg @ \varg$   & \verb|\forall| \\
   \hline
   Reversed E              & existential quant.     & $\exists\varg@ \varg$    & \verb|\exists| \\
   \hline
   $\exists$ subscript $1$ & unique existence       & $\exists_1\varg @ \varg$ & \verb|\exists_1| \\
   \hline
   Pertinence              & set membership         & $\varg\in\varg$          & \verb|\in| \\
   \hline
   Math. \verb|\times|     & cartesian product      & $\varg\cross\varg$       & \verb|\cross| \\
   \hline
   Inverted solidus        & schema hiding          & $\varg\hide\varg$        & \verb|\hide| \\
   \hline
   Upwards harpoon         & schema projection      & $\varg\project\varg$     & \verb|\project| \\
   \hline
   Big fat semicolon       & schema composition     & $\varg\semi\varg$        & \verb|\semi| \\
   \hline
   Double greater than     & schema piping          & $\varg\pipe\varg$        & \verb|\pipe| \\
   \hline
   Big fat colon           & typechecked term       & $\varg\typecolon\varg$  & \verb|\typecolon| \\
   \hline
\end{tabular}
\caption{Core symbols \smallcaption{A.2.4.4}}\label{tbl:symbol-core}
\end{table}
%
The ampersand (\verb|\&|) is needed in (the not so used) mutually
recursive free types. Its syntax is described in~\cite[8.2]{isoz},
whereas its semantics is given in~\cite[14.2.3.1]{isoz}.
The fat \verb|\spot| also makes \verb|@| active in math mode
so that it gets the right \verb|\mathrel| spacing.
Wedge and Vee are the AMS terms for the logical operators.

Other core symbols, such as ``$/$'', ``$;$'', ``$:$'', ``$,$'',
``$.$'', ``$+$'', ``$=$'', \textit{etc.}, are typeset and spaced
just as in \LaTeX. The symbol for schema projection ($\project$,
\verb|\project|) is reused for sequence filtering in the
toolkit defined in \seqtkfile (see Table~\ref{tbl:symbol-toolkit-seq} in Section~\ref{sec:symbol-toolkit-seq}).
Also, the symbol for schema composition ($\semi$, \LaTeX{} \verb|\semi|, and Unicode character U+$2A1F$)
is very similar (but slightly bigger) than the symbol for relational
composition ($\comp$,~\verb|\comp|,~U+$2A3E$). Type checked markup is
usually given with a big fat colon beside it ($\typecolon$,~\verb|\typecolon|,~U+$2982$).
It is a binary operator with the expression in one side and its type on the other.
%
%\hline
%Solidus           & substitution   & $/$   & \verb|/| \\
%\hline
%Double horz. bars & equality       & $=$   & \verb|=| \\
%\hline
%Colon             & type expr. sep. & $:$ & \verb|:| \\
%\hline
%Semicolon         & var decl. sep.  & $;$ & \verb|;| \\
%\hline
%Comma             & var list sep.   & $,$ & \verb|,| \\
%\hline
%Dot               & tuple/bind sel. & $.$ & \verb|.| \\
%\hline
%Plus              & integer sum     & $+$ & \verb|+| \\
%\hline

Note that spacing with \LaTeX{} infix binary mathematical operators are rendered
differently in the presence of new lines in between them.
%
\begin{demo}
\begin{verbatim}
\begin{zed}
  A ~~==~~ S \cup T \\        % usual
  B ~~==~~ S \cup \\ \t2 T \\ % spacing after \cup symbol
  C ~~==~~ S \cup{} \\ \t2 T  % correction
\end{zed}
\end{verbatim}
\gives
\begin{zed}
   A ~~==~~ S \cup T \\          % usual
   B ~~==~~ S \cup \\ \t2 T \\   % bad spacing after \cup symbol
   C ~~==~~ S \cup{} \\ \t2 T    % correction
\end{zed}
\end{demo}
%
So, when breaking lines
near such operators, one need to add the usual \LaTeX{} marker for such situations,
as illustrated below (see~\cite[p.525, Table~8.7]{latexcomp} for more details), new
lines may change the spacing behaviour of infix binary mathematical operators,
as the example above shows.

\subsection{Toolkit symbols}\label{sec:symbol-toolkit}

This section introduces all the characters used within \stdtkfile, as mentioned
in~\cite[Appendix~B]{isoz}. It has been divided in subsections according to the various
Z sections defined in the Standard.

\subsubsection{\preludefile{} and Z keywords}\label{sec:symbol-toolkit-prelude}

The \texttt{prelude} section is an implicit parent of every other section.
It assists in defining the meaning of number literal expressions~\cite[12.2.6.9]{isoz}
and the list arguments of operator templates~\cite[12.2.12]{isoz} via syntactic transformation
rules. In Table~\ref{tbl:symbol-toolkit-prelude}, we present the list of symbols
and Z keywords (and their fixture) defined in the prelude with markup directives.
%
\begin{table}[ht]
\centering
\begin{tabular}{|l|l|l|l|}
   \hline
   \textbf{Description} & \textbf{Role} & \textbf{Rendering} & \textbf{\LaTeX} \\
   \hline
   Z section marker     & prefix keyword      & $\SECTION\varg$                 & \verb|\SECTION| \\
   \hline
   Z section parent     & infix keyword       & $\parents\varg$                 & \verb|\parents| \\
   \hline
   Conditional          & prefix keyword      & $\IF\varg$                      & \verb|\IF| \\
   \hline
   Conditional          & infix keyword       & $\varg\THEN\varg$               & \verb|\THEN| \\
   \hline
   Conditional          & infix keyword       & $\varg\ELSE\varg$               & \verb|\ELSE| \\
   \hline
   Let definition       & prefix keyword      & $\LET\varg==\varg @ \varg$      & \verb|\LET| \\
   \hline
   Application expr.    & prefix op. template & $\function\varg~\varg$          & \verb|\function| \\
   \hline
   Relational pred.     & prefix op. template & $\relation\varg~\varg$          & \verb|\relation| \\
   \hline
   Generic expr. inst.  & prefix op. template & $\generic\varg~\varg$           & \verb|\generic| \\
   \hline
   Left associative     & infix op. template  & $\leftassoc$                    & \verb|\leftassoc| \\
   \hline
   Right associative    & infix op. template  & $\rightassoc$                   & \verb|\rightassoc| \\
   \hline
   Schema precondition  & prefix keyword      & $\pre\varg$                     & \verb|\pre| \\
   \hline
   List of arguments    & infix op. template  & $\Rightarrow\listarg\Leftarrow$ & \verb|\listarg| \\
   \hline
   Variable argument    & infix op. template  & $\Rightarrow\varg\Leftarrow$    & \verb|\varg| \\
   \hline
   Boolean truth       & ordinary name   & $\true$                         & \verb|\true| \\
   \hline
   Boolean falsehood   & ordinary name   & $\false$                        & \verb|\false| \\
   \hline
\end{tabular}
\caption{\preludefile symbols \smallcaption{A.2.4, B.2}}\label{tbl:symbol-toolkit-prelude}
\end{table}
%
Z sections enable the user to define self contained named modules with (non cyclic)
parent relationships given as a (possibly empty) list of section names. Conditional
($\IF-\THEN-\ELSE$) allows one to test a predicate which yields an expression depending whether
the predicate is $true$ or $false$. Let definitions ($\LET$) allow local variable scoping for expressions.

Operator templates~\cite[C.4.13]{isoz} have syntactic significance only:~they tell
the reader how to interpret the template associativity, and how it is rendered
as prefix, infix, posfix or nofix. There are three categories of operator templates the
user can define:~\verb|\function|, for application expressions as \textit{e.g.,}
\[ S \cup T = (\_~\cup\_)~(S, T) \]
\verb|\relation|, for relational predicates as \textit{e.g.,}
\[ S \subseteq T = (S,T) (\_~\subseteq\_) \]
and \verb|\generic|, for generic instantiation of expressions as \textit{e.g.,}
\[ X \rel Y,\t1 \emptyset[\nat] \]
Application expressions (\verb|\function|) are used for both fixed (as pre, in, or pos fixed)
function operator application (\textit{e.g.,}~infix $S \cup T$), and as its equivalent
(\textit{e.g.,}~nofix $(\_~\cup\_)~(S, T)$) version.
Relational (or membership) predicates (\verb|\relation|) are used for both set membership
(\textit{e.g.,}~\mbox{$x \in S$}), equality (\textit{e.g.,}~\mbox{$S = T$}), and as an
operator that is a predicate (\textit{e.g.,}~\mbox{$S \subseteq T$}).
Generic instantiation expressions are used for generic operator application
as in when building relation (\mbox{$X \rel Y$}) or function (\mbox{$X \fun Y$}) spaces.

Furthermore, all infix \verb|\function| and \verb|\generic| operator templates must have two explicit
declarations:~one for their binding power, which is as a natural number (the higher the number tighter the
precedence);~and one for their (left or right) associativity. They are used to resolve operator
precedences. For instance, \mbox{$S \cup T \cap R = (S \cup (T \cap R))$} because $\cap$ binds tighter
than $\cup$ (see binding powers in Table~\ref{tbl:symbol-toolkit-set} below). Relational predicates
and prefix, posfix and nofix function and generic operators do not have precedence or associativity
explicitly given. Examples of this notation can be found in~\cite[Appendix~B]{isoz}, and are highlighted
in the \textbf{Role} column in the tables below for each operator template defined in the standard toolkits.
When the binding powers are the same, the given associativity is used to resolve the precedence.
For instance, set intersection and set difference have the same binding power ($30$), but are
both left-associative, hence $S \cap T \setminus R = (S \cap T) \setminus R$ as the left-associativeness
of set intersection gives it priority over set difference. Finally, if within the same section (and
all its parent sections) there are two operator templates with the same binding power (even if different
kinds, say one \verb|\function| and one \verb|\generic|), but different associativity, a parsing error
is raised since precedence cannot be decided. For instance, if we have a section with \texttt{set\_toolkit}
as its parent, and we define a new \verb|\function| operator template with binding power $30$
and as right associative, a parsing error is raised, as it is not possible to decide its
precedence (\textit{i.e.,} it conflicts with the operator template definition for $\cup$).

Note that generic operator templates, such as finite subsets $(\finset~\_)$ and total functions $(\_~\fun\_)$,
are not to be confused with a generic reference expression instantiation, such as empty sets ($\emptyset[\nat]$),
which is not given as an operator template, but rather a reference name. Moreover, when generic
references are instantiated by the typechecker they are implicit ($\emptyset$), whereas when given by
the user they are explicit ($\emptyset[\nat]$---the empty set of natural numbers).

When defining operator templates, we could have single arguments (\verb|\varg|) as in the
definition of set union ($\varg\cup\varg$, \verb|\varg \cup \varg|) at \settkfile, or variable/list
arguments (\verb|\listarg|) as in the definition of sequence display ($\langle \listarg \rangle$,
\verb|\langle \listarg \rangle|) at \seqtkfile.


Other Z style packages allow room for a keyword \verb|\inrel|, which could be used for changing
the fixture of relations that were not defined as operator templates. For instance, suppose
$R \in X \rel Y$, $x \in X$, and $y \in Y$. As $R$ is not an operator template, the usual
way of relating $x$ and $y$ to $R$ would be either ``\mbox{$(x,y) \in R$}'' or ``\mbox{$(x \mapsto y \in R)$}''.
With the \verb|\inrel| keyword, one was allowed to say ``\mbox{$(x~\inrel{R}~y)$}'' (\verb|(x~\inrel{R}~y)|).
Nevertheless, such feature is not part of the Z Standard, hence not amenable to parsing, and thus
not supported in \cztstylefile.

Additionally, we add two special ``keywords'' as $\true$ (\verb|\true|) and $\false$ (\verb|\false|)
to represent boolean values at the level of the logic, rather than as predicates $true$ (\verb|$true$|)
and $false$ (\verb|$false$|). This is used in the Z logic of the Z Standard. It can also be used in the
definition of a boolean free type in a user toolkit. This serves to illustrate how one can make use of
Z markup directives once again.
%
\begin{demo}
\begin{verbatim}
% AMS black board B
% \bool is already defined in czt.sty just like
% \newcommand{\bool}{\zordop{\mathbb B}}

% Note the markup directives are needed for parsing
% since they are not present in any standard toolkit.
%%Zchar \bool U-0001D539
%%Zword \true True
%%Zword \false False
\begin{zed}
   \bool ::= \false | \true
\end{zed}
\end{verbatim}
\gives
% AMS black board B
% \bool is already defined in czt.sty just like
% \newcommand{\bool}{\zordop{\mathbb B}}

% Note the markup directives are needed for parsing
% since they are not present in any standard toolkit.
%%Zchar \bool U-0001D539
%%Zword \true True
%%Zword \false False
\begin{zed}
   \bool ::= \false | \true
\end{zed}
\end{demo}
%
Apart from typesetting purposes, logic boolean values can be used, for instance,
to use Z as a meta-language to specify the semantics of other languages~\cite{circussem}.

\subsubsection{\settkfile}\label{sec:symbol-toolkit-set}

The \texttt{set\_toolkit} defines symbols for what a relation is, and operators about sets and finite sets.
In Table~\ref{tbl:symbol-toolkit-set}, we present the list its symbols.
The \textbf{Role} column contains the details for each operator template, or ``\textit{XXX name}''
when the symbol is not an operator but a name, where the \textit{XXX} determines its fixture.
Infix function and generic operator templates have their binding power given as numbers, and
associativity given as either LA (left-associative) or RA (right-associative). Non-infix operator templates have
their type and fixture given. For ease of reference, we also add the \verb|\varg| arguments to the
\LaTeX{} rendering column (but not the verbatim \LaTeX{} itself for clarity).
%
\begin{table}[ht]
\centering
\begin{tabular}{|l|l|c|l|}
   \hline
   \textbf{Description} & \textbf{Role} & \textbf{Rendering} & \textbf{\LaTeX} \\
   \hline
   Relation space      & generic $5$ RA   & $\varg\rel\varg$      & \verb|\rel| \\
   \hline
   Function space      & generic $5$ RA   & $\varg\fun\varg$      & \verb|\fun| \\
   \hline
   Not set member      & infix relation   & $\varg\notin\varg$    & \verb|\notin| \\
   \hline
   Inequality          & infix relation   & $\varg\neq\varg$      & \verb|\neq| \\
   \hline
   Empty set           & nofix name       & $\emptyset$           & \verb|\emptyset| \\
   \hline
   Subset              & infix relation   & $\varg\subseteq\varg$ & \verb|\subseteq| \\
   \hline
   Proper subset       & infix relation   & $\varg\subset\varg$   & \verb|\subset| \\
   \hline
   Non-empty sets      & prefix name      & $\power_1\varg$       & \verb|\power_1| \\
   \hline
   Set union           & function $30$ LA & $\varg\cup\varg$      & \verb|\cup| \\
   \hline
   Set intersection    & function $40$ LA & $\varg\cap\varg$      & \verb|\cap| \\
   \hline
   Set difference      & function $30$ LA & $\varg\setminus\varg$ & \verb|\setminus| \\
   \hline
   Set symmetric diff. & function $25$ LA & $\varg\symdiff\varg$  & \verb|\symdiff| \\
   \hline
   Generalised union   & prefix name      & $\bigcup\varg$        & \verb|\bigcup| \\
   \hline
   Generalised intersection & prefix name & $\bigcap\varg$        & \verb|\bigcap| \\
   \hline
   Finite sets         & prefix generic   & $\finset\varg$        & \verb|\finset| \\
   \hline
   Non empty $\finset$ & prefix generic   & $\finset_1\varg$      & \verb|\finset_1| \\
   \hline
\end{tabular}
\caption{\settkfile symbols \smallcaption{A.2.5.1, B.3, B.4}}\label{tbl:symbol-toolkit-set}
\end{table}

The empty set symbol within the usual \LaTeX{} distribution (as found in file
\emfile{fontmath.ltx} with font encoding \verb|OMS/cmsy/m/n| and hex number \verb|"3B|)
is slightly different from the mathematical empty set symbol, which is present in the AMS font.
Because of this, when using \cztstylefile, one can access the original empty set symbol with
\verb|\mathemptyset|, which is rendered in \LaTeX{} as $\mathemptyset$.

\subsubsection{\reltkfile}\label{sec:symbol-toolkit-relation}

The \texttt{relation\_toolkit} has \texttt{set\_toolkit} as its parent and
defines symbols for:~maplets;~domain and range;~relational and functional
composition;~domain and range restriction and substraction;~relational
inversion and overriding;~and transitive and reflexive transitive closures over relations.
In Table~\ref{tbl:symbol-toolkit-relation}, we present its symbols.
%
\begin{table}[ht]
\centering
\begin{tabular}{|l|l|c|l|}
   \hline
   \textbf{Description} & \textbf{Role} & \textbf{Rendering} & \textbf{\LaTeX} \\
   \hline
   Binary tuple projection & ordinary name     & $first$             & \verb|first~\varg| \\
   \hline
   Binary tuple projection & ordinary name     & $second$            & \verb|second~\varg| \\
   \hline
   Relation maplet         & function $10$ LA  & $\varg\mapsto\varg$ & \verb|\mapsto| \\
   \hline
   Domain of relation     & prefix name        & $\dom\varg$         & \verb|\dom| \\
   \hline
   Range  of relation     & prefix name        & $\ran\varg$         & \verb|\ran| \\
   \hline
   Identity relation      & prefix generic     & $\id\varg$          & \verb|\id| \\
   \hline
   Relational composition & function $40$ LA   & $\varg\comp\varg$   & \verb|\comp| \\
   \hline
   Functional composition & function $40$ LA   & $\varg\circ\varg$   & \verb|\circ| \\
   \hline
   Domain restriction     & function $65$ LA   & $\varg\dres\varg$   & \verb|\dres| \\
   \hline
   Range restriction      & function $60$ LA   & $\varg\rres\varg$   & \verb|\rres| \\
   \hline
   Domain subtraction     & function $65$ LA   & $\varg\ndres\varg$  & \verb|\ndres| \\
   \hline
   Range subtraction      & function $60$ LA   & $\varg\nrres\varg$  & \verb|\nrres| \\
   \hline
   Relational inversion   & prefix function    & $\varg\inv$         & \verb|\inv| \\
   \hline
   Relational image left  & mixfix function    & $\varg\limg$        & \verb|\limg| \\
   \hline
   Relational image right & mixfix function    & $\varg\rimg$        & \verb|\rimg| \\
   \hline
   Overriding             & function $50$ LA   & $\varg\oplus\varg$  & \verb|\oplus| \\
   \hline
   Transitive closure     & posfix function    & $\varg\plus$        & \verb|\plus| \\
   \hline
   Reflexive $(\_~\plus)$ & posfix function    & $\varg\star$        & \verb|\star| \\
   \hline
\end{tabular}
\caption{\reltkfile symbols \smallcaption{A.2.5.2, B.5}}\label{tbl:symbol-toolkit-relation}
\end{table}

This toolkit defines tuple projection functions that do not use markup directives and
are not given as operator templates, hence have no special \LaTeX{} markup associated
with them. Despite this fact, the usual \LaTeX{} rendering is (historically)
given as if they were Z keywords. To achieve this effect, however, the user need define his own
``special'' rendering for that markup. For instance, $first$ and $second$, which project the first and second elements
of a given binary tuple, are defined with ordinary names (\textit{i.e.,}~no markup directive)
in \reltkfile. So, some users prefer to have keyword-like typesetting, which can be done as \verb|\zkeyword{first}|
($\zkeyword{first}$). Unfortunately, this is no longer parseable, since \verb|\zkeyword| is not part of the Z lexis,
but rather a \LaTeX{} rendering markup. Nevertheless, if the user still wants to keep a nice
\LaTeX{} rendering, she could just define the appropriate \LaTeX{} command as an alternative markup for the name in
question through markup directives. For our example, to have ``$first$'' typeset like a keyword ($\zkeyword{first}$),
one should add the following markup directive and new \LaTeX{} command:
%
\newcommand{\first}{\zpreop{\zkeyword{first}}}
\begin{verbatim}
\newcommand{\first}{\zpreop{\zkeyword{first}}}
%%Zpreword \first first
\end{verbatim}
%
The markup directive will tell the parser to treat the command \verb|\first| as
the string \verb|first|, which is loaded from \reltkfile. Then \LaTeX{} can now
render \verb|\first| as desired ($\zkeyword{first}$). Furthermore, if the user
wants to keep both choices conditional to the package option \texttt{tkkeyword},
one can just use the \verb|\ztoolkit{first}|, instead. With current option, it
typesets as $\ztoolkit{first}$. Finally, note that, since $\first$ a prefix word,
we also wrap it with a \verb|\zpreop|. This way proper spacing is added, and one
does not need to typeset it as \verb|$\first~x$| ($\first~x$) but just
\verb|$\first x$| ($\first x$), as opposed to the mandatory hard space in
\verb|$first~x$| ($first~x$) to avoid the wrong spacing from \verb|$first x$| ($first x$).

The Z Standard also leaves room for mixfix (mixed fixture) operator templates, although those
are more rarely used. One such operator is used for the definition of relational imagine as
%
\begin{verbatim}
%%Zinchar   \limg U+2987
%%Zpostchar \rimg U+2988
\begin{zed}
\function (\_ \limg \_ \rimg)
\end{zed}
\end{verbatim}
%
So, each bracketing symbol is treated with a different fixture. That is, $\limg$ is treated
as an infix operator, whereas $\rimg$ is treated as a posfixed one. This combination makes
the relational image mixfix operator template as defined above.

The AMS/Lucida bright font(s) already define(s) the \verb|\star| symbol as ``$\mathstar$''
(\textit{e.g.,}~\texttt{msam10}, hex-number \verb|"46|), rather than the ``$\star$'' we want.
Because of this, when using \cztstylefile, one can access the original AMS/Lucida bright star
symbol with  the \verb|\mathstar| command, which is rendered as ``$\mathstar$''.

For relational inverse (\verb|R\inv|), the Z Standard does not specify it with superscripting
word glues~\cite[A.2.4.3]{isoz}. Thus, its rendering is ``$R\inv$'', and it should not be superscripted
as in ``$R^{\inv}$'', despite this being more common. This may perhaps be a Z Standard typo.

\subsubsection{\fcntkfile}\label{sec:symbol-toolkit-function}

The \texttt{function\_toolkit} has \texttt{relation\_toolkit} as its parent and
defines symbols for generic operator templates representing the various subsets
of function spaces, and a few relational predicates for sets. In
Table~\ref{tbl:symbol-toolkit-function}, we present its list of symbols.
%
\begin{table}[ht]
\centering
\begin{tabular}{|l|l|c|l|}
   \hline
   \textbf{Description} & \textbf{Role} & \textbf{Rendering} & \textbf{\LaTeX} \\
   \hline
   Partial function        & generic $5$ RA   & $\varg\pfun\varg$   & \verb|\pfun| \\
   \hline
   Partial injection       & generic $5$ RA   & $\varg\pinj\varg$   & \verb|\pinj| \\
   \hline
   Injection               & generic $5$ RA   & $\varg\inj\varg$    & \verb|\inj| \\
   \hline
   Partial surjection      & generic $5$ RA   & $\varg\psurj\varg$  & \verb|\psurj| \\
   \hline
   Surjection              & generic $5$ RA   & $\varg\surj\varg$   & \verb|\surj| \\
   \hline
   Bijection               & generic $5$ RA   & $\varg\bij\varg$    & \verb|\bij| \\
   \hline
   Finite partial function & generic $5$ RA   & $\varg\ffun\varg$   & \verb|\ffun| \\
   \hline
   Finite partial injection & generic $5$ RA  & $\varg\finj\varg$   & \verb|\finj| \\
   \hline
   Disjoint sets            & prefix relation & $\disjoint\varg$    & \verb|\disjoint| \\
   \hline
   Set partitioning         & infix relation  & $\varg\partition\varg$ & \verb|\partition| \\
   \hline
\end{tabular}
\caption{\fcntkfile symbols \smallcaption{A.2.5.3, B.6}}\label{tbl:symbol-toolkit-function}
\end{table}
%
Lucida Bright fonts render some of these symbols differently, if (and when) loaded.

Disjointness of a relation states that a set of sets has no overlapping elements
(\textit{i.e.,}~their pairwise set intersection is empty), whereas partitioning
represents a disjoint set of sets that covers the whole elements of the set's type
(\textit{i.e.,}~the generalised union of all sets being disjoint represents the whole type).

\subsubsection{\numtkfile}\label{sec:symbol-toolkit-number}

The \texttt{number\_toolkit} defines symbols for integer arithmetic.
In Table~\ref{tbl:symbol-toolkit-number}, we present its list symbols.
Note that summation is defined as an operator template in \preludefile,
but most of its properties are defined in \numtkfile, hence we left it here.
Subtraction is defined in terms of summation and unary negation
(\textit{e.g.,}~$\negate\varg$, \verb|\negate|).
%
\begin{table}[ht]
\centering
\begin{tabular}{|l|l|c|l|}
   \hline
   \textbf{Description} & \textbf{Role} & \textbf{Rendering} & \textbf{\LaTeX} \\
   \hline
   $\nat$ successor function & prefix function  & $succ\varg$      & \verb|succ \varg| \\
   \hline
   Integers                  & ordinary name    & $\num$           & \verb|\num| \\
   \hline
   Arithmetic negation       & prefix function  & $\negate\varg$   & \verb|\negate| \\
   \hline
   Subtraction               & function $30$ LA & $\varg-\varg$    & \verb|-| \\
   \hline
   Summation                 & function $30$ LA & $\varg+\varg$    & \verb|+| \\
   \hline
   Less-than equal-to        & infix relation   & $\varg\leq\varg$ & \verb|\leq| \\
   \hline
   Less-than                 & infix relation   & $\varg<\varg$    & \verb|<| \\
   \hline
   Greater-than equal-to     & infix relation   & $\varg\geq\varg$ & \verb|\geq| \\
   \hline
   Greater-than              & infix relation   & $\varg>\varg$    & \verb|>| \\
   \hline
   Non empty $\nat$          & prefix name      & $\nat_1$         & \verb'\nat_1' \\
   \hline
   Non empty $\num$          & prefix name      & $\num_1$         & \verb'\num_1' \\
   \hline
   Multiplication            & function $40$ LA & $\varg*\varg$    & \verb|*| \\
   \hline
   Integer division          & function $40$ LA & $\varg\div\varg$ & \verb|\div| \\
   \hline
   Integer modulus           & function $40$ LA & $\varg\mod\varg$ & \verb|\mod| \\
   \hline
\end{tabular}
\caption{\numtkfile symbols \smallcaption{A.2.5.4, B.7}}\label{tbl:symbol-toolkit-number}
\end{table}

Like what happened in \reltkfile, where definitions were given without markup
directives, in \numtkfile, the successor function for natural numbers ($succ$)
is also defined without markup directives, yet one may be familiar with its
specialised rendering as $\ztoolkit{succ}$ (\verb|\ztoolkit{succ}|). This is slightly
different from $first$ and $second$ from \reltkfile, as $succ$ is defined as an operator
template in \numtkfile, hence the \verb|\varg| on its description in Table~\ref{tbl:symbol-toolkit-number}.

The division symbol within the usual \LaTeX{} distribution (\emfile{fontmath.ltx}
with font encoding \verb|OMS/cmsy/m/n| and hex value \verb|"04|) is different from the Z
integer division symbol, which is given as a Z toolkit word (\verb|\ztoolkit{div}|) in \cztstylefile. To access
the original definition, one should use \verb|\mathdiv| ($\mathdiv$), instead.

\subsubsection{\seqtkfile}\label{sec:symbol-toolkit-seq}

The \texttt{sequence\_toolkit} has \texttt{function\_toolkit} and \texttt{number\_toolkit}
as its parents and defines range, relational iteration, set cardinality, min/max, and
finite sequences and its operators. In Table~\ref{tbl:symbol-toolkit-seq}, we present
its list of symbols.
%
\begin{table}[ht]
\centering
\begin{tabular}{|l|l|c|l|}
   \hline
   \textbf{Description} & \textbf{Role} & \textbf{Rendering} & \textbf{\LaTeX} \\
   \hline
   Number range          & function $20$ LA & $\varg\upto\varg$          & \verb|\upto| \\
   \hline
   Iteration             & ordinary name    & $iter\varg\varg$           & \verb|iter| \\
   \hline
   Iteration             & prefix function  & $(\varg~^{~\varg~})$       & \verb|\varg~^{~\varg~}| \\
   \hline
   $\finset$ cardinality & prefix function  & $\#\varg$                  & \verb|\#| \\
   \hline
   Minimum               & prefix function  & $min\varg$                 & \verb|min~\varg| \\
   \hline
   Maximum               & prefix function  & $max\varg$                 & \verb|max~\varg| \\
   \hline
   Finite seq.           & prefix generic   & $\seq\varg$                & \verb|\seq| \\
   \hline
   Non empty $\seq$      & prefix name      & $\seq_1\varg$              & \verb|\seq_1| \\
   \hline
   Injective seq.        & prefix generic   & $\iseq\varg$               & \verb|\iseq| \\
   \hline
   Seq. brackets         & mixfix function  & $\langle \listarg \rangle$ & {\footnotesize \verb|\langle \listarg \rangle|} \\
   \hline
   Concatenation         & function $30$ LA & $\varg\cat\varg$           & \verb|\cat| \\
   \hline
   Seq. reverse          & ordinary name    & $rev\varg$                 & \verb|rev~\varg| \\
   \hline
   Seq. head             & ordinary name    & $head\varg$                & \verb|head~\varg| \\
   \hline
   Seq. last             & ordinary name    & $last\varg$                & \verb|last~\varg| \\
   \hline
   Seq. tail             & ordinary name    & $tail\varg$                & \verb|tail~\varg| \\
   \hline
   Seq. front            & ordinary name    & $front\varg$               & \verb|front~\varg| \\
   \hline
   Seq. re-indexing      & ordinary name    & $squash\varg$              & \verb|squash~\varg| \\
   \hline
   Seq. extraction       & function $45$ LA & $\varg\extract\varg$       & \verb|\extract| \\
   \hline
   Seq. filtering        & function $40$ LA & $\varg\filter\varg$        & \verb|\filter| \\
   \hline
   Seq. prefix           & prefix relation  & $\varg\prefix\varg$        & \verb|\prefix| \\
   \hline
   Seq. suffix           & prefix relation  & $\varg\suffix\varg$        & \verb|\suffix| \\
   \hline
   Seq. infix            & prefix relation  & $\varg\infix\varg$         & \verb|\infix| \\
   \hline
   Dist. concat.         & ordinary name    & $\dcat$                    & \verb|\dcat| \\
   \hline
\end{tabular}
\caption{\seqtkfile symbols \smallcaption{A.2.5.5, B.8}}\label{tbl:symbol-toolkit-seq}
\end{table}

In \seqtkfile, few ordinary names or operator templates without markup directive
also are typeset as keywords. They are:~relation iteration ($\ztoolkit{iter}~R~i$)
and its superscript version ($R~^{~i~}$);~minimum ($min$) and maximum ($max$) of a set of
numbers;~sequence $rev$erse, $head$, $last$, $tail$, $front$, and $squash$;~and
distributed concatenation ($\dcat$). It is questionable if some of them should be
made prefix function operator templates in the Z Standard. Note that, as these are
ordinary names, no special \LaTeX{} spacing scheme is in place. Thus, although
not explicitly required by the CZT tools, to properly render these names,
a hard space is required in order to separate them from their arguments (\textit{e.g.,}~``$rev~s$'', \verb|$rev~s$|).
Otherwise, \LaTeX{} will typeset them as a single word (\textit{e.g.,}~``$rev s$'', \verb|$rev s$|).
Again, if wanted, markup directives with corresponding \LaTeX{} macros as \verb|\ztoolkit| can be added.

\subsubsection{\stdtkfile}\label{sec:symbol-toolkit-std}

The \texttt{standard\_toolkit} has \texttt{sequence\_toolkit} as its parent
and defines nothing. It is the Z section implicitly inherited if no $\SECTION$
keyword is present within a given file. Such files have so-called ``implicit''
sections, where the implicit section is named as the file (without its extension),
where the \texttt{standard} \texttt{\_toolkit} is its parent~\cite[B.9]{isoz}.

\section{Z-\LaTeX{} environments}\label{sec:latex-env}

In Table~\ref{tbl:latex-env}, we describe all the Z-\LaTeX{} environments used to
typeset the various Z paragraphs, such as:~Z section headers containing the section
name and its (optional, possibly empty, list of) parents;~horizontal paragraphs like given sets,
operator templates, free types, horizontal schemas, and unnamed conjectures;~named
conjecture paragraphs;~axiomatic and generic axiomatic definitions;~and schema and generic
schema definitions. In many of these paragraphs, the \verb|\where| keyword is used
to separate the declaration part from the predicate part. The \texttt{ENDCHAR}
is used to mark the end of all Z paragraphs within the Unicode character stream.
%
\begin{table}[ht]
\centering
\begin{tabular}{|l|l|l|}
    \hline
    \textbf{Description} & \textbf{Markup} & \textbf{\LaTeX} \\
    \hline
    Section header        & \texttt{ZEDCHAR}              & \verb|\begin{zsection}|      \\
    \hline
    Horizonal paragraph   & \texttt{ZEDCHAR}              & \verb|\begin{zed}|           \\
    \hline
    Named conjecture      & \texttt{ZEDCHAR}              & \verb|\begin{theorem}{thm}|  \\
    \hline
    Axiomatic definition  & \texttt{AXCHAR}               & \verb|\begin{axdef}|         \\
    \hline
    Generic axdef         & \texttt{AXCHAR GENCHAR}       & \verb|\begin{gendef}|        \\
    \hline
    Schema definition     & \texttt{SCHCHAR}              & \verb|\begin{schema}{S}|     \\
    \hline
    Generic schema        & \texttt{SCHCHAR GENCHAR}      & \verb|\begin{schema}{S}[X]|  \\
    \hline
    Declaration separator & \verb'~|~', or \verb|~\mid~|  & \verb|\where|                \\
    \hline
    End of all Z paras    & \texttt{ENDCHAR}              & \verb|\end{XXX}|             \\
    \hline
\end{tabular}
\caption{Z-\LaTeX{} environments \smallcaption{A.2.6, A.2.7}}\label{tbl:latex-env}
\end{table}

Only material within Z paragraphs and \LaTeX{} markup directives are treated by CZT tools
as part of a formal Z specification. Insofar as tools are concerned, everything else
(\textit{e.g.,}~plain text, \LaTeX{} comments, other \LaTeX{} commands, \textit{etc.})
is treated as a Z narrative paragraph, which can contain arbitrary text.

To illustrate these boxes, we introduce a few Z paragraphs below. They are inspired in
Mike Spivey's guide to Z-\LaTeX{} markup (\textit{i.e.,}~\texttt{zed2e.tex}). Firstly,
we add a series of horizonal paragraphs.
%
\begin{demo}
\begin{verbatim}
\begin{zed}
   % Hard spaces (~) are optional below. They were
   % added for (personal) aesthetic reasons.
   [Set]
   \also    % small vertical space
   List ~~::=~~ leaf | const \ldata List \rdata \\
   Sch  ~~==~~ [~ x, y: \nat | x > y ~] \\
   Sch2 ~~==~~ Sch \land [~ z: \num ~]
\end{zed}
\end{verbatim}
\gives
\begin{zed}
   [Set]
   \also    % small vertical space
   List ::= leaf | const \ldata List \rdata \\
   Sch  ~~==~~ [~ x, y: \nat | x > y ~] \\
   Sch2 ~~==~~ Sch \land [~ z: \num ~]
\end{zed}
\end{demo}
%
Next, we typeset an axiomatic definition.
%
\begin{demo}
\begin{verbatim}
\begin{axdef}
   f, g: \power~\nat \fun (\num \cross \seq~\arithmos)
\where
\zbreak     % may not break, depends on page placement
   \forall S, T: \power~\nat | f~S \subseteq g~S @ \\
        \t1 first~(f~S) < \#~(g~S).2
\end{axdef}
\end{verbatim}
\gives
\begin{axdef}
   f, g: \power~\nat \fun (\num \cross \seq~\arithmos)
\where
\zbreak     % may not break, depends on placement

   \forall S, T: \power~\nat | f~S \subseteq g~S @ \\
        \t1 \, \! \: \, first~(f~S) < \#~(g~S).2
\end{axdef}
\end{demo}
After that, we have a simple vertical schema.
%
\begin{demo}
\begin{verbatim}
\begin{schema}{Test}
   x, y: \nat; S, T: \power_1~\nat
\where
   x > y \\
   S \subset T \\
\znewpage           % certainly breaks
   x \neq y = 0
   \Also            % medium vspace
   x \in S \land y \notin T
\end{schema}
\end{verbatim}
\gives
\begin{schema}{Test}
   x, y: \nat; S, T: \power_1~\nat
\where
   x > y \\
   S \subset T \\
\znewpage           % certainly breaks
   x \neq y = 0
   \Also            % medium vspace
   x \in S \land y \notin T
\end{schema}
\end{demo}
%
Below we typeset a generic axiomatic definition.
%
\begin{demo}
\begin{verbatim}
\begin{gendef}[X, Y]
   S, T: \power~(X \cross Y)
\where
   S \subseteq T
   \ALSO           % big vertical space

   \exists U: \power~(X \cross Y) \spot \\
        \t2 U \subset (S \cup T)
\end{gendef}
\end{verbatim}
\gives
\begin{gendef}[X, Y]
   S, T: \power~(X \cross Y)
\where
   S \subseteq T
   \ALSO           % big vertical space

   \exists U: \power~(X \cross Y) \spot U \subset (S \cup T)
\end{gendef}
\end{demo}
%
And finally, a generic schema.
%
\begin{demo}
\begin{verbatim}
\begin{schema}{GenTest}[X]
   a: X; b: \power~X
\where
   a \in b
\end{schema}
\end{verbatim}
\gives
\begin{schema}{GenTest}[X]
   a: X; b: \power~X
\where
   a \in b
\end{schema}
\end{demo}
%
For schemas without names, which are not recognised by the parser, one could
use the \verb|\begin{plainschema}| environment.
%
\begin{demo}
\begin{verbatim}
\begin{plainschema}
    x, y: \nat
 \where
    x = y
\end{plainschema}
\end{verbatim}
\gives
\begin{plainschema}
    x, y: \nat
 \where
    x = y
\end{plainschema}
\end{demo}
%
Finally, stared versions of the usual Z environments can be used to
typeset Z-\LaTeX, but having its text ignored by the CZT tools as a narrative paragraph.
%
\begin{demo}
\begin{verbatim}
\begin{zed*}
   [NotParsed]
\end{zed*}
\begin{axdef*}
   a : \arithmos
\end{axdef*}
\begin{gendef*}[X]
   x: X
\end{gendef*}
\begin{schema*}{NotParsed}[X]
   x, y: X
\where
   x > y
\end{schema*}
\end{verbatim}
\gives
\begin{zed*}
   [NotParsed]
\end{zed*}
\begin{axdef*}
   a : \arithmos
\end{axdef*}
\begin{gendef*}[X]
   x: X
\end{gendef*}
\begin{schema*}{NotParsed}[X]
   x, y: X
\where
   x > y
\end{schema*}
\end{demo}

\newpage

After we have done that, let us test trailing spaces after Z paragraph environments are not
affecting L/R mode indentation spacing, a known problem in some old Z-\LaTeX{} style files.
Say, let us define a new operator template as a prefix function. For that we also add,
together with the operator definition, its (Z) \LaTeX{} markup directive and associated
\LaTeX{} markup command as a \verb|\ztoolkit|.
%
\begin{demo}
\begin{verbatim}
% Unicode markup in markup directive is the "text" to use
\newcommand{\test}{\ztoolkit{test}}

\begin{zed}
   %%Zword \test test
   \function (\test \_)
\end{zed}
Now some text to see if paragraph mode indentation is right.
\[ \forall x: \nat_1 @ x > 0 \]
What about with math display environments?
All seems okay.
\end{verbatim}
\gives
% Unicode markup in markup directive is the "text" to use
\newcommand{\test}{\ztoolkit{test}}

\begin{zed}
   %%Zword \test test
   \function (\test \_)
\end{zed}
Now some text to see if paragraph mode indentation is right.
\[ \forall x: \nat_1 @ x > 0 \]
What about with math display environments? All seems okay.
\end{demo}
%
Finally, let us test the named conjecture environment.
%
\begin{demo}
\begin{verbatim}
\begin{theorem}{Thm1}
   \forall x: \nat @ x \geq 0
\end{theorem}
\end{verbatim}
\gives
\begin{theorem}{Thm1}
   \forall x: \nat @ x \geq 0
\end{theorem}
\end{demo}
%
Unfortunately, I could not find a way to make named conjectures
colourful, whenever colour is enabled in Z math mode.

\section{Controlling definition counting}\label{sec:cztcount}

One interesting new feature added to the style file is the ability
to (\textbf{roughly}) count the number of each Z definition specified
in a given portion of \LaTeX. The counting is controlled by few
package options, which determine if counting should be globally,
by Chapter, or by Section. The actual counters can also be directly
accessed and/or changed, where automatic counting can also be switched
on and off if one wants tighter control over when what should be counted.

There are three package options associated with counting:
%
\begin{itemize}
   \item \texttt{cntglobally}:~counts definitions globally;
   \item \texttt{cntbychapter}:~definition count reset by each Chapter;
   \item \texttt{cntbysection}:~definition count reset by each Section.
\end{itemize}
%
For the local counters by Chapter and Section we also have a global counter.
This way one can keep track of both local and global amounts. Another important
command is the boolean flag \verb|\CountDefs|. It determines whether counting
switched on (\verb|\CountDefstrue|) or off (\verb|CountDefsfalse|). It can be
used to selectively count parts of a specification.

Another command to use is \verb|\ZDeclSummary|. It typesets a table
with seven lines and two or three columns, depending on the counting
option chosen. The lines are:~the column header; the number of
unboxed items, which represent given sets, free types, (named) conjectures,
operator templates, and abbreviations (\textit{e.g.,}~constant definitions
or horizontal schemas);~the number of axiomatic and generic axiomatic
definitions;~the number of schemas and generic schemas;~and finally the
total number of declarations. The columns are:~each line description;~and
the totals per line, where local totals are added in the case of section
and chapter options had been chosen. Moreover, each number given in this
table is represented internally by a \LaTeX{} counter, which the user can
both access and interfere with if needed.

For example, for this file, where we have chosen the counting by section option,
issuing \verb|\ZDeclSummary| produces the contents shown in Table~\ref{tbl:zdecl:\thesection}.
%
\ZDeclSummary
%
Note that, since this section has no specification, nothing is counted locally.
Moreover, to refer to the table \verb|\ZDeclSummary| typesets, a unique label
is specified for it, and one can use the command \verb|\ref{tbl:zdecl:\thesection}|.
For Chapter counting, one needs to use \verb|\thechapter| in the place of \verb|\thesection|,
and just \verb|\ref{tbl:zdecl:global}| when counting globally. These labels \texttt{should}
not be defined by users, or a \textit{multiply defined label} \LaTeX{} warning will arise.

More advanced usage of the counting facility can be achieved by directly
influencing the way the counters work, either by accessing their value,
or interfering in the way counting works. One should not do that usually,
unless extending the counting facilities to include other pieces of information,
say, number of examples or proofs given by a specification.

\begin{table}[ht]
    \centering
    \begin{tabular}{|l|l|l|}
        \hline
        \textbf{Declarations}   & \textbf{Local counter name}    & \textbf{Global counter name} \\
        \hline
        Unboxed items           & \texttt{cntZunboxed}           & \texttt{cntZtotunboxed} \\
        \hline
        Axiomatic definitions   & \texttt{cntZaxdef}             & \texttt{cntZtotaxdef} \\
        \hline
        Generic axiomatic defs. & \texttt{cntZgendef}            & \texttt{cntZtotgendef} \\
        \hline
        Schemas                 & \texttt{cntZschema}            & \texttt{cntZtotschema} \\
        \hline
        Generic schemas         & \texttt{cntZgenschema}         & \texttt{cntZtotgenschema} \\
        \hline
        Total                   & \texttt{cntZdecl}              & \texttt{cntZtotdecl} \\
        \hline
    \end{tabular}
    \caption{Z definition counters}\label{tbl:cztcount}
\end{table}
%
When counting locally by Chapter or Section, the local counter names are reset according
to \LaTeX's Chapter or Section counters change. The equivalent global counters are not
reset and serve to accumulate the global totals, also included in the information. Note
that the local total counter (\texttt{cntZtotdecl}) simply sums the local counters, whereas
the global total counter (\texttt{cntZdecl}) sums the global counters. Finally, when
counting globally, only the global counters are.

One can access the value of each of these counters using the usual \LaTeX{} commands
to manipulate counters, such as:~\verb|\stepcounter{name}|, which increases counter
\texttt{name} by one;~and \verb|\thename|, which accesses the \verb|\arabic{name}| counter value.
Moreover, when counting by Chapter or Section, we redefine the contents of \verb|\thename| value
for all local counters to include the Chapter or Section where the local counter belongs to. For
example, for the value of \texttt{cntZunboxed}, the result of \verb|\thecntZunboxed| is
``\thecntZunboxed''. One can easily redefine such command for whatever else is
appropriated.

Finally, there are two more low-level \LaTeX{} commands used to control counting. The
command \verb|\@countZpara| is called within the various Z-\LaTeX{} environments to
actually perform the counters update. A single command within all environment simplify
the counting code. If \verb|\CountDefs| is true, then when \verb|\@countZpara| is expanded,
the appropriate \LaTeX{} counters described above are updated depending on the value of a
low-level \TeX{} register \verb|\@zcountingwhat|. That is, the \TeX{} register determines
what is being counted. The values used are:
%
\begin{itemize}
   \item \verb|\@zcountingwhat=0|:~counts unboxed items affecting \texttt{cntZunboxed};
   \item \verb|\@zcountingwhat=1|:~counts axiomatic defs. affecting \texttt{cntZaxdef};
   \item \verb|\@zcountingwhat=2|:~counts generic axiomatic defs. affecting \texttt{cntZgendef};
   \item \verb|\@zcountingwhat=3|:~counts schemas affecting \texttt{cntZschema};
   \item \verb|\@zcountingwhat=4|:~counts generic schemas affecting \texttt{cntZgenschema};
   \item \verb|\@zcountingwhat=99|:~counts nothing affecting no counter.
\end{itemize}
%
One use for \verb|\@zcountingwhat=99| is, for instance, to avoid counting non-parseable
definitions, such as the stared versions of the Z-\LaTeX{} environments (\textit{e.g.,}~\verb|\begin{zed*}|).
Package developers extending this counting facility can use this \TeX{} register and
other low-level commands to count their own definitions. When extending Z-\LaTeX{} environments,
package developers are advised to look into the more detailed documentation of \cztstylefile,
which explains more of the low-level \LaTeX/\TeX{} commands available.

Axiomatic and generic definitions are counted within each \verb|\begin{axdef}| and \verb|\begin{gendef}| commands,
whereas schemas and generic definitions are counted within each \verb|\begin{schema}{name}| and
\verb|\begin{schema}[X]{name}|, respectively. Unboxed items are introduced through the \verb|\begin{zed}| or
\verb|\begin{theorem}{name}| commands. Nevertheless, counting must also take place when Z-\LaTeX{} new lines are introduced,
as they represent declaration separators. That is, every \verb|\\| or \verb|\also| command within \verb|\begin{zed}| will
also increase the counter. The same applies for the variations of the \verb|\also| command (\textit{i.e.,}~\verb|\Also| and
\verb|\ALSO|). Note that, since the Z Standard allows multiple new lines to separate commands (\textit{i.e.,}~more than one
\verb|\\| between commands), this could create a discrepancy between the number of definitions counted, and the actual number
specified. Such discrepancy may only occur for unboxed items, whenever their specification has multiple line separators.

\section{Extra macros and commands from \cztstylefile}\label{sec:cztsty}

There are a few extra macros the user may refer to when extending the
\cztstylefile, or adding her own markup directives. They are summarised
in Table~\ref{tbl:cztsty-extra}.
%
\begin{table}[ht]
\centering
\begin{tabular}{|l|l|}
    \hline
    \textbf{Description} & \textbf{\LaTeX} \\
    \hline
    \cztstylefile{} version       & \verb|\fileversion|      \\
    \hline
    \cztstylefile{} date          & \verb|\filedate|           \\
    \hline
    \cztstylefile{} description   & \verb|\filedesc|  \\
    \hline
    \cztstylefile{} file name     & \verb|\cztstylefile| \\
    \hline
    Prefix operators               & \verb|\zpreop{XXX}|         \\
    \hline
    Posfix operators        & \verb|\zpostop{XXX}|        \\
    \hline
    Binary operators               & \verb|\zbinop{XXX}|     \\
    \hline
    Relational operators       & \verb|\zrelop{XXX}|  \\
    \hline
    Ordinary operators    & \verb|\zordop{XXX}|  \\
    \hline
    Big symbol            & \verb|\zbig{XXX}| \\
    \hline
    Bigger symbol         & \verb|\zBig{XXX}| \\
    \hline
    Even bigger symbol    & \verb|\zBIG{XXX}| \\
    \hline
    Smaller symbol        & \verb|\zSmall{XXX}| \\
    \hline
    Even smaller symbol   & \verb|\zsmall{XXX}| \\
    \hline
    Partial symbol        & \verb|\p{XXX}| \\
    \hline
    Finite symbol         & \verb|\f{XXX}| \\
    \hline
    Block alignment env.  & {\small \verb|\begin{zblock}\end{zblock}|} \\
    \hline
\end{tabular}
\caption{Extra \LaTeX{} macros in \smallcaption{\cztstylefile}}\label{tbl:cztsty-extra}
\end{table}
%
File version, date, and description are simple strings with
information about \cztstylefile. The various operator wrappers
are used to tell \LaTeX{} how spaces for some particular markup
should be treated. They follow the usual \LaTeX{} mathematical
operators spacing rules (see~\cite[p.~525, Table~8.7]{latexcomp}).
Some symbols can be increased or decreased relative to their base symbol.
For instance, the symbol for schema composition ($\semi$) is the \verb|\zBig|
version of the symbol for relational composition ($\comp$).
Similarly, partial function spaces ($\pfun$) are just the \verb|\p|
version of total functions ($\fun$). Finally, block alignment can be
used so that the treatment of new line within the block adds extra
spacing just after the new line.

\section{Conclusions and acknowledgements}\label{sec:conclusions}

In this document, we presented a guide to typesetting ISO Standard Z~\cite{isoz} in \LaTeX{}
when typeset using the \cztstylefile. The document is divided to mirror the
Standard as much as possible. This style file is the result of merging, filtering,
and removing definitions from various other style files, such as \texttt{oz.sty},
\texttt{soz.sty}, \texttt{zed-csp.sty}, \texttt{zed.sty}, \texttt{fuzz.sty},
\texttt{z-eves.sty}, and so on.

The main design decision behind this document follows CZT guideline that
``what you type is what you model''. That is, the document ``as-is'' becomes
the source Standard Z (\LaTeX) specification to be processed by tools. Other
design decisions included:~i)~keep the style file as minimal, simple, and
consistent as possible;~ii)~document and acknowledge macro definition choices
and their origin (when different);~iii)~normalise definitions for
consistency;~iv)~complete missing cases with either normative rules from
the Standard or using common sense;~v)~keep the style file well documented,
but not verbose;~and vi)~follow order of definitions from Z Standard document.

As the \cztstylefile{} may be used by both language extensions and \LaTeX{} users,
we also provided and explained a series of useful macros for \LaTeX{} rendering
that bare no relation with the Standard or the tools. They are useful for \LaTeX{}
typesetting only, and are explained in Section~\ref{sec:intro-cztopt}, and
Section~\ref{sec:cztsty}.

We tried to present, as exhaustively as possible, the use of every one
of such commands with \LaTeX{} markup typeset in verbatim mode for
clarity and reference. We summarise them all in an Appendix below.
More details can be found at the \texttt{czt.dvi} file generated with
the \texttt{docstrip} utility on the \texttt{czt.dtx} document from
the CZT distribution.

Finally, the author would like to thank \textit{QinetiQ Malvern} in the
UK for its long term support for the development of formal verification
tools here at York. Also, the work to prepare this document and its companion
style file benefited immensely by the good work of previous package
builders for Z, namely Sebastian Rahtz (Object Z, \emfile{oz.sty}),
Mike Spivey (ZRM and Fuzz, \emfile{zed.sty, fuzz.sty}), Jim Davies
(ZRM and CSP$_M$, \emfile{zed-csp.sty}), Ian Toyn (Standard Z Editor,
\emfile{ltcadiz.sty, soz.sty}), and Mark Utting (original CZT style
based on \emfile{oz.sty}, \emfile{czt.sty}). Moreover, I would like
to thank all the people in the \texttt{czt-devel} mailing list for
their helpful comments on my many questions. Finally, I need to
thank my York colleagues Jim Woodcock and Juan Perna for many helpful
discussions about tool design and \LaTeX{} typesetting.

\section{Features left out}\label{sec:todo}

There were several features left out from the various packages we got
inspiration from which might be of good use in typesetting \LaTeX{}
specifications, as shown below in Table~\ref{tbl:todo}.
%
\begin{table}[ht]
\centering
\begin{tabular}{|l|l|l|}
    \hline
    \textbf{Description}      & \textbf{Source}               & \textbf{\LaTeX} \\
    \hline
    Multiple column math mode & \texttt{oz.sty}              & \verb|\begin{sidebyside}|      \\
    \hline
    Comment in math mode      & \texttt{oz.sty}              & \verb|\comment{XXX}|           \\
    \hline
    indented new lines alignment  & \texttt{oz.sty}              & various \\
    \hline
    Tabular alignment math mode & \texttt{zed.sty}               & \verb|\begin{syntax}|         \\
    \hline
    Hand written proofs         & \texttt{zed.sty}       & \verb|\begin{argue}|        \\
    \hline
    Inference rules         & \texttt{zed.sty}              & \verb|\begin{infrule}|     \\
    \hline
    Mechanical proof scripts & \texttt{z-eves.sty}      & \verb|\begin{zproof}|  \\
    \hline
    Labelled predicates      & \texttt{z-eves.sty}      & \verb|\Label{XXX}|  \\
    \hline
    Various new line alignment & \texttt{z-eves.sty}   & \verb|\+, \-, \\| \\
    \hline
\end{tabular}
\caption{Some \LaTeX{} macros left out from other style files}\label{tbl:todo}
\end{table}
\newpage % add here to avoid table in same page as reference card
%
Although some of them could be introduced without problem as \textit{e.g.,}
\begin{verbatim}
    \begin{sidebyside}...\end{sidebyside}
\end{verbatim}
for most others the trouble is their presence within the Z-\LaTeX{} lexis.
That is, their presence would be detected by the parser as an error, hence they were left out.

Finally, note that the Z Standard does not define a toolkit for multi sets also known as bags.
That is despite the fact most Z tools do, and the symbols are well known from Spivey's guide~\cite{zrm}.
Eventually, we should have in CZT extra toolkits from either known sources and rigorous experiments.

\newpage
\appendix
\section{Reference card}\label{app:ref-card}

\begin{multicols}{2}
\setcounter{secnumdepth}{2}
\zedindent=0.2\zedindent

\subsection{Letters}
\vspace*{-0.5ex}

\subsubsection{Special Greek}
\vspace*{-2.5ex}

%\begin{symbols}[??] = controls amount of vertical space added
\begin{symbols}
 \Delta  & \verb|\Delta| \\
 \Xi     & \verb|\Xi| \\
 \theta  & \verb|\theta| \\
 \lambda & \verb|\lambda| \\
 \mu & \verb|\mu| \\
 \Phi    & \verb|\Phi|
\end{symbols}

\subsubsection{Small Greek}
\vspace*{-2.5ex}

\begin{symbols}
\alpha   & \verb|\alpha| \\
\beta    & \verb|\beta| \\
\gamma   & \verb|\gamma| \\
\delta   & \verb|\delta| \\
\epsilon & \verb|\epsilon| \\
\zeta    & \verb|\zeta| \\
\eta     & \verb|\eta| \\
\iota    & \verb|\iota| \\
\kappa   & \verb|\kappa| \\
\nu      & \verb|\nu| \\
\xi      & \verb|\xi| \\
\pi      & \verb|\pi| \\
\rho     & \verb|\rho| \\
\sigma   & \verb|\sigma| \\
\tau     & \verb|\tau| \\
\upsilon & \verb|\upsilon| \\
\phi     & \verb|\phi| \\
\chi     & \verb|\chi| \\
\psi     & \verb|\psi| \\
\omega   & \verb|\omega|
\end{symbols}

\subsubsection{Capital Greek}
\vspace*{-2.5ex}

\begin{symbols}
\Gamma   & \verb|\Gamma| \\
\Theta   & \verb|\Theta| \\
\Lambda  & \verb|\Lambda| \\
\Pi      & \verb|\Pi| \\
\Sigma   & \verb|\Sigma| \\
\Upsilon & \verb|\Upsilon| \\
\Phi     & \verb|\Phi| \\
\Psi     & \verb|\Psi| \\
\Omega   & \verb|\Omega|
\end{symbols}

\subsection{Special Z characters}
\vspace*{-0.5ex}

\subsubsection{Stroke chars}
\vspace*{-2.5ex}
\begin{symbols}
  ' & \verb|'| \\
  ! & \verb|!| \\
  ? & \verb|?|
\end{symbols}

\subsubsection{Brackets}
\vspace*{-2.5ex}
\begin{symbols}
( & \verb|(| \\
) & \verb|)| \\
[ & \verb|[| \\
] & \verb|]| \\
\{\mbox{\textvisiblespace} & \verb|\{~| \\
\mbox{\textvisiblespace}\} & \verb|~\}| \\
\lblot & \verb|\lblot| \\
\rblot & \verb|\rblot| \\
\ldata& \verb|\ldata| \\
\rdata & \verb|\rdata| \\
\Rightarrow\_\Leftarrow & \verb|\_| \\
\Rightarrow\varg\Leftarrow & \verb|\varg|
\end{symbols}

\subsubsection{Spacing}
\vspace*{-2.5ex}

\begin{symbols}
\Rightarrow ~ \Leftarrow  & \verb|~| \\
\Rightarrow\ \Leftarrow & \verb|\|\textvisiblespace \\
\Rightarrow \, \Leftarrow & \verb|\,| \\
\Rightarrow \: \Leftarrow & \verb|\:| \\
\Rightarrow \; \Leftarrow & \verb|\;| \\
\Rightarrow \t1 \Leftarrow &\t2 \verb|\t1| \\
\Rightarrow \t2 \Leftarrow &\t2 \verb|\t2| \\
\mbox{new line}     &\t2 \verb|\\| \\
\mbox{small vspace} &\t2 \verb|\also| \\
\mbox{med. vspace}  &\t2 \verb|\Also| \\
\mbox{big. vspace}  &\t2 \verb|\ALSO| \\
\mbox{small break}  &\t2 \verb|\zbreak| \\
\mbox{med. break}   &\t2 \verb|\zBreak| \\
\mbox{big. break}   &\t2 \verb|\ZBREAK| \\
\mbox{new page}     &\t2 \verb|\znewpage|
\end{symbols}

\subsection{Z Notation}
\vspace*{-0.5ex}

\subsubsection{Logic}
\vspace*{-2.5ex}

\begin{symbols}
%Core symbols
%@, \spot          & \verb|@|, \verb|\spot| \\
%\varg\&\varg      & \verb|\&| \\
\vdash P           &\t1 \verb|\vdash P| \\
P \land Q          &\t1 \verb|P \land| \\
P \lor Q           &\t1 \verb|P \lor| \\
P \implies Q       &\t1 \verb|P \implies| \\
P \iff Q           &\t1 \verb|P \iff| \\
\lnot P            &\t1 \verb|\lnot P| \\
\forall x: T @ P   &\t1 \verb|\forall x: T @ P| \\
\exists x: T @ P   &\t1 \verb|\exists x: T @ P| \\
\exists_1 x: T @ P &\t1 \verb|\exists_1 x: T @ P| \\
x \in S            &\t1 \verb|x \in S| \\
X \cross Y         &\t1 \verb|X \cross Y| \\
S \hide (x)        &\t1 \verb|S \hide (x)| \\
S \project T       &\t1 \verb|S \project T| \\
S \semi T          &\t1 \verb|S \semi T| \\
S \pipe T          &\t1 \verb|S \pipe T| \\
E \typecolon T    &\t1 \verb|E \typecolon T| \\
\true              &\t1 \verb|\true| \\
\false             &\t1 \verb|\false| \\
\bool              &\t1 \verb|\bool|
\end{symbols}

\subsubsection{Z keywords}
\vspace*{-2.5ex}

\begin{symbols}
%Prelude symbols
\SECTION name           &\t2 \verb|\SECTION name| \\
\parents s1, s2         &\t2 \verb|\parents s1, s2| \\
\IF P\THEN E1\ELSE E2   &\t3 \verb|\IF P \THEN E1 \ELSE E2| \\
\LET x==y @ P           &\t3 \verb|\LET x == y @ P| \\
\pre S                  &\t3 \verb|\pre S| \\
\function               &\t3 \verb|\function|\\
\relation               &\t3 \verb|\relation| \\
\generic                &\t3 \verb|\generic| \\
\leftassoc              &\t3 \verb|\leftassoc|\\
\rightassoc             &\t3 \verb|\rightassoc|
\end{symbols}

\subsection{Mathematical toolkits}
\vspace*{-0.5ex}

\subsubsection{Set toolkit}
\vspace*{-2.5ex}
\begin{symbols}
X \rel Y      &\t1 \verb|X \rel Y| \\
X \fun Y      &\t1 \verb|X \fun Y| \\
x \notin S    &\t1 \verb|x \notin S| \\
x \neq y      &\t1 \verb|x \neq y| \\
\emptyset     &\t1 \verb|\emptyset| \\
S \subseteq T &\t1 \verb|S \subseteq T| \\
S \subset T   &\t1 \verb|S \subset T| \\
\power  X     &\t1 \verb|\power X| \\
\power_1 X    &\t1 \verb|\power_1 X| \\
S \cup T      &\t1 \verb|S \cup T| \\
S \cap T      &\t1 \verb|S \cap T| \\
S \setminus T &\t1 \verb|S \setminus T| \\
S \symdiff T  &\t1 \verb|S \symdiff T| \\
\bigcup SS    &\t1 \verb|\bigcup SS| \\
\bigcap SS    &\t1 \verb|\bigcap SS| \\
\finset X     &\t1 \verb|\finset X| \\
\finset_1 X   &\t1 \verb|\finset_1 X|
\end{symbols}

\subsubsection{Relation toolkit}
\vspace*{-2.5ex}

\begin{symbols}
first~t         &\t1 \verb|first~t| \\
second~t        &\t1 \verb|second~t| \\
x \mapsto y     &\t1 \verb|\mapsto| \\
\dom R          &\t1 \verb|\dom| \\
\ran R          &\t1 \verb|\ran| \\
\id R           &\t1 \verb|\id| \\
R \comp S       &\t1 \verb|R \comp S| \\
R \circ S       &\t1 \verb|R \circ S| \\
R \dres S       &\t1 \verb|R \dres S| \\
R \rres S       &\t1 \verb|R \rres S| \\
R \ndres S      &\t1 \verb|R \ndres S| \\
R \nrres S      &\t1 \verb|R \nrres S| \\
R \inv          &\t1 \verb|R \inv| \\
R \limg S \rimg &\t1 \verb|R \limg S \rimg| \\
R \rimg         &\t1 \verb|R \rimg| \\
R \oplus S      &\t1 \verb|R \oplus S| \\
R \plus         &\t1 \verb|R \plus| \\
R \star         &\t1 \verb|R \star|
\end{symbols}

\subsubsection{Function toolkit}
\vspace*{-2.5ex}
\begin{symbols}
X \pfun Y   &\t2 \verb|X \pfun Y| \\
X \pinj Y   &\t2 \verb|X \pinj Y| \\
X \inj Y    &\t2 \verb|X \inj Y| \\
X \psurj Y  &\t2 \verb|X \psurj Y| \\
X \surj Y   &\t2 \verb|X \surj Y| \\
X \bij Y    &\t2 \verb|X \bij Y| \\
X \ffun Y   &\t2 \verb|X \ffun Y| \\
X \finj Y   &\t2 \verb|X \finj Y| \\
\disjoint S &\t2 \verb|\disjoint S| \\
{S \partition T} &\t2 \verb|S \partition T|
\end{symbols}

\subsubsection{Number toolkit}
\vspace*{-2.5ex}
\begin{symbols}
\arithmos &\t1 \verb|\arithmos| \\
\num      &\t1 \verb|\num| \\
\num_1    &\t1 \verb|\num_1| \\
\nat      &\t1 \verb|\nat| \\
\nat_1    &\t1 \verb|\nat_1| \\
\rat      &\t1 \verb|\rat| \\
\real     &\t1 \verb|\real| \\
succ~n    &\t1 \verb|succ~n| \\
\negate x &\t1 \verb|\negate x| \\
x-y       &\t1 \verb|x - y| \\
x+y       &\t1 \verb|x + y| \\
x\leq y   &\t1 \verb|x \leq y| \\
x<y       &\t1 \verb|x < y| \\
x\geq y   &\t1 \verb|x \geq y| \\
x>y       &\t1 \verb|x > y| \\
x *y      &\t1 \verb|x * y| \\
x \div y  &\t1 \verb|x \div y| \\
x \mod y  &\t1 \verb|x \mod y|
\end{symbols}

\subsubsection{Sequence toolkit}
\vspace*{-2.5ex}

\begin{symbols}
x \upto y            &\t1 \verb|x \upto y| \\
iter~R~i             &\t1 \verb|iter~R~i| \\
(R~^{~i~})           &\t1 \verb|R~^{~i~}| \\
\#~S                 &\t1 \verb|\#~S| \\
min~S                &\t1 \verb|min~S| \\
max~S                &\t1 \verb|max~S| \\
\seq X               &\t1 \verb|\seq X| \\
\seq_1 X             &\t1 \verb|\seq_1 X| \\
\iseq X              &\t1 \verb|\iseq X| \\
\langle x, y \rangle &\t1 \verb|\langle x, y \rangle| \\
s \cat t             &\t1 \verb|\cat| \\
rev~s                &\t1 \verb|rev~s| \\
head~s               &\t1 \verb|head~s| \\
last~s               &\t1 \verb|last~s| \\
tail~s               &\t1 \verb|tail~s| \\
front~s              &\t1 \verb|front~s| \\
squash~s             &\t1 \verb|squash~s| \\
S \extract s         &\t1 \verb|S \extract s| \\
s \filter S          &\t1 \verb|s \filter S| \\
s \prefix t          &\t1 \verb|s \prefix t| \\
s \suffix t          &\t1 \verb|s \suffix t| \\
s \infix t           &\t1 \verb|s \infix t| \\
\dcat~ss             &\t1 \verb|\dcat~ss|
\end{symbols}

\end{multicols}

\newpage
\bibliographystyle{plain}
\bibliography{czt-guide}

\end{document}  
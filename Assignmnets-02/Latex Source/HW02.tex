\documentclass{article}
\usepackage{czt} 
\usepackage{xepersian}  
\newcommand{\sectionbreak}{\clearpage}
\settextfont[Scale=1.2]{BZAR.TTF}

\title{\lr{Formal Specification and Verification of Programs}}
\author{\lr{2nd Assignment Solutions}\\
\lr{Mohammad Hossein Khoshechin - 99210164}\\
\lr{Group 2}}

\begin{document}

\maketitle

\section*{تمرین ۱ : بازی \lr{X-O}}
بازی \lr{X-O} یک بازی دو نفره است که بر روی یک جدول انجام می شود . یک بازیکن به طور قردادی برای علامت زدن خانه های جدول از نماد \lr{X} استفاده می کند و بازیکن دیگر از نماد \lr{O} استفاده می کند.هر بازیکن زمانی که نوبتش رسید می تواند تنها در یکی از خانه های خالی جدول ، نمادش را قرار دهد . قرار داد می کنیم که در زمان شروع بازی ،  نوبت متعلق به بازیکن \lr{X} می باشد . هر بازیکن بعد از اینکه در نوبت خود یک خانه را علامت زد ، نوبت به بازیکن دیگر داده می شود. هر زمان که سه خانه از جدول به شکل سطری ،  ستونی و یا قطری ، تنها با استفاده از یک نماد پر شود آنگاه بازیکنی که با آن نماد بازی میکرده ،  برنده است . زمانی که تمام خانه های جدول پر شود اما هیچ یک از سطرها ،  ستون ها و قطرها ،  تنها با یک نماد پر نشده باشد آنگاه بازی مساوی می شود.
\\
\begin{zed}
Index~ == ~\{a : \nat | a < 3\}~\\
Cell~ == ~Index \cross Index\\
State == ~empty|x|o\\
Turn == x|o\\
Message == Cell\_Filled | Action\_Faild | Xwin | Owin | noWinner
\end{zed}

\begin{schema}{GameTable}
table : Cell \fun State\\
pturn : Turn
\end{schema}

\begin{schema}{GameTableInit}
GameTable'
\where
\forall c:Cell @ table'~c = empty\\
pturn' = x
\end{schema}
\\
تا به اینجا \lr{state} را تعریف کردیم که دارای یک جدول است. هر خانه از جدول هم به یک وضعیت مپ شده است در حالت \lr{initial} هم تمام خانه ها را \lr{empty} در نظر میگیرم و نوبت را به بازیکن \lr{x} میدهیم.
\\
\\
حال دو تا \lr{Schema Operator} مینویسیم برای بازی کردن بازکن \lr{x} و بازیکن \lr{o} . در هر کدام یک متغیر \lr{Cell} به عنوان وروردی میگیریم و یک پیام \lr{output} هم به عنوان خروجی قرار میدهیم.
\\
\\
\begin{schema}{Oplay}
\Delta GameTable\\
location? : Cell\\
output! : Message
\where
pturn = o\\
table~ ~location? = empty\\
table' = table \oplus \{~ location? \mapsto o ~\}\\
output! = Cell\_Filled\\
pturn' = x
\end{schema}
\\
\\
\begin{schema}{Xplay}
\Delta GameTable\\
location? : Cell\\
output! : Message
\where
pturn = x\\
table~ ~location? = empty\\
table' = table \oplus \{~ location? \mapsto x ~\}\\
output! = Cell\_Filled\\
pturn' = o
\end{schema}
\\
حال باید این نکته را در نظر بگیریم که ممکن است زمانی که یک بازیکن میخواهد خانه ای از جدول را پر کند ، آن خانه خالی نباشد یا نوبتش نباشد. پس باید یک \lr{Schema Operator} هم برای هر دو بازی کن لحاظ کنیم .
\\

\begin{schema}{OplayFaild}
\Xi GameTable\\
location? : Cell\\
output! : Message
\where
pturn = x\\
table~ ~location? \neq empty\\
output! = Action\_Faild
\end{schema}

\begin{schema}{XplayFaild}
\Xi GameTable\\
location? : Cell\\
output! : Message
\where
pturn = o\\
table~ ~location? \neq empty\\
output! = Action\_Faild
\end{schema}
\\
حال با \lr{disjunction} کردن \lr{Schema Operator} ها ، \lr{Schema Operator} مربوط به حرکت هر دو بازیکن را تعریف می کنیم :

\begin{zed}
MoveX == Xplay \lor XplayFaild\\
MoveO == Oplay \lor OplayFaild
\end{zed}
\\
حال با نوشتن چند \lr{Schema Operator} ، شرط خاتمه بازی را تعریف میکنیم . برای این کار باید سه تا \lr{Schema Operator} بنویسیم   یکی باری برد بازیکن \lr{x} ، یکی برای برد بازیکن \lr{o} و دیگری برای حالت تساوی.
\\

\begin{schema}{Xwin}
\Xi GameTable\\
output! : Message
\where
\exists i_1 , i_2 , i_3 , j_1 , j_2 , j_3 : Index @\\
\t1 (~(i_1=i_2) \land (i_2=i_3) \land (j_2 = j_1+1) \land (j_3 = j_2+1) \land\\
\t1 \t1 \t1 \t1 (table(\{~ (i_1,j_1) , (i_2,j_2) , (i_3,j_3) ~\}) = \{~ x ~\})~) \lor\\
\t1 (~(j_1=j_2) \land (j_2=j_3) \land (i_2 = i_1+1) \land (i_3 = i_2+1) \land\\
\t1 \t1 \t1 \t1 (table(\{~ (i_1,j_1) , (i_2,j_2) , (i_3,j_3) ~\}) = \{~ x ~\})~) \lor\\
\t1 (~(i_1,j_1)=(0,0) \land (i_2,j_2)=(1,1) \land (i_3,j_3)=(2,2) \land\\
\t1 \t1 \t1 \t1 (table(\{~ (i_1,j_1) , (i_2,j_2) , (i_3,j_3) ~\}) = \{~ x ~\})~) \lor\\
\t1 (~(i_1,j_1)=(0,2) \land (i_2,j_2)=(1,1) \land (i_3,j_3)=(0,2) \land\\
\t1 \t1 \t1 \t1 (table(\{~ (i_1,j_1) , (i_2,j_2) , (i_3,j_3) ~\}) = \{~ x ~\})~)\\
output! = Xwin
\end{schema}

\begin{schema}{Owin}
\Xi GameTable\\
output! : Message
\where
\exists i_1 , i_2 , i_3 , j_1 , j_2 , j_3 : Index @\\
\t1 (~(i_1=i_2) \land (i_2=i_3) \land (j_2 = j_1+1) \land (j_3 = j_2+1) \land\\
\t1 \t1 \t1 \t1 (table(\{~ (i_1,j_1) , (i_2,j_2) , (i_3,j_3) ~\}) = \{~ o ~\})~) \lor\\
\t1 (~(j_1=j_2) \land (j_2=j_3) \land (i_2 = i_1+1) \land (i_3 = i_2+1) \land\\
\t1 \t1 \t1 \t1 (table(\{~ (i_1,j_1) , (i_2,j_2) , (i_3,j_3) ~\}) = \{~ o ~\})~) \lor\\
\t1 (~(i_1,j_1)=(0,0) \land (i_2,j_2)=(1,1) \land (i_3,j_3)=(2,2) \land\\
\t1 \t1 \t1 \t1 (table(\{~ (i_1,j_1) , (i_2,j_2) , (i_3,j_3) ~\}) = \{~ o ~\})~) \lor\\
\t1 (~(i_1,j_1)=(0,2) \land (i_2,j_2)=(1,1) \land (i_3,j_3)=(0,2) \land\\
\t1 \t1 \t1 \t1 (table(\{~ (i_1,j_1) , (i_2,j_2) , (i_3,j_3) ~\}) = \{~ o ~\})~)\\
output! = Owin
\end{schema}

\begin{schema}{WithDraw}
\Xi GameTable\\
output! : Message
\where
\forall c: Cell @ table~ ~c \neq empty\\
\forall i_1 , i_2 , i_3 , j_1 , j_2 , j_3 : Index @\\
\t1 (~((i_1=i_2) \land (i_2=i_3) \land (j_2 = j_1+1) \land (j_3 = j_2+1)) \implies\\
\t1 \t1 \t1 \t1 (\#table(\{~ (i_1,j_1) , (i_2,j_2) , (i_3,j_3) ~\}) \neq 1)~) \land\\
\t1 (~((j_1=j_2) \land (j_2=j_3) \land (i_2 = i_1+1) \land (i_3 = i_2+1)) \implies\\
\t1 \t1 \t1 \t1 (\#table(\{~ (i_1,j_1) , (i_2,j_2) , (i_3,j_3) ~\}) \neq 1)~) \land\\
\t1 (~((i_1,j_1)=(0,0) \land (i_2,j_2)=(1,1) \land (i_3,j_3)=(2,2)) \implies\\
\t1 \t1 \t1 \t1 (\#table(\{~ (i_1,j_1) , (i_2,j_2) , (i_3,j_3) ~\}) \neq 1)~) \land\\
\t1 (~((i_1,j_1)=(0,2) \land (i_2,j_2)=(1,1) \land (i_3,j_3)=(0,2)) \implies\\
\t1 \t1 \t1 \t1 (\#table(\{~ (i_1,j_1) , (i_2,j_2) , (i_3,j_3) ~\}) \neq 1)~) \\
output! = noWinner
\end{schema}

\begin{zed}
End == Xwin \lor Owin \lor WithDraw
\end{zed}

\begin{zed}
XOGame == MoveX \lor MoveO \lor End
\end{zed}

\section*{تمرین ۲ : بازی \lr{pac-man}}
بازی پکمن به این صورت است که در یک جدول انجام میشود . بعضی از خانه های جدول دیوار \lr{wall} هستند و بعضی دیگر مسیر \lr{path}. \lr{Agent} های بازی به شکل پکمن ، روح ها و دانه ها هستند . ما به عنوان بازیکن تنها هدایت پکمن را بر عهده داریم . طبیعتا پکمن تنها در خانه هایی که مسیر هستند می تواند حرکت کند و نمی تواند وارد خانه هایی شود که دیوار هستند . پکمن زمانی که در یک خانه وارد میشود اگر در آن خانه دانه باشد آن را می خورد و اگر روح باشد کشته می شود و بازی تمام می شود . زمانی برنده می شویم که پکمن تمام دانه های موجود در جدول را بخورد. دانه ها تنها در خانه هایی که دیوار نیستند قرار داده می شوند . حرکت روح ها در این سناریو به طور تصادفی است و هدفمند نیست . روح ها نیز مانند پکمن نمی توانند وارد خانه هایی شوند که دیوار است . دیوار ها باید طوری در جدول قرار گیرند که بین هر دو خانه ای که در جدول دیوار نیستند(\lr{path} هستند) ، مسیری از خانه های \lr{path} وجود داشته باشد.
\\ 
\begin{zed}
Cell~ == ~\nat \cross \nat\\
State == ~wall | path\\
Message == win | defeated\\
Move == UP | DOWN | LEFT | RIGHT
\end{zed}

\begin{schema}{GameTable}
table : Cell \fun State\\
pacman : Cell\\
ghosts : \power~Cell\\
seeds : \power~Cell\\
length : \nat\\
width : \nat
\where
\forall i,j : \nat | (i,j) \in dom ~ table @ i < width \land j < length\\
\exists c_1,c_2,c_3 : Cell @ (c_1 \neq c_2) \land (c_2 \neq c_3 ) \land (table(\{~ c_1 , c_2 , c_3 ~\}) = \{~ path ~\}) \land\\
\t1 \t1 \{~ c_1 , c_2 , c_3 ~\} \subset dom~~ table
\forall c: Cell | c \in seeds @ table~~c = path\\
\#gosts \ge 1\\
\#seeds \ge 1\\
table ~ pacman = path\\
pacman \in dom~~table\\
seeds \subset dom~~table\\
ghosts \subset dom~~table\\
table(ghosts) = \{~ path ~\}\\
table(seeds) = \{~ path ~\}\\
\forall c_1,c_2:Cell| c_1 \in dom~~ table \land c_2 \in dom~~ table \land c_1 \neq c_2@ (table(\{~ c_1 , c_2~\}) = \{~ path ~\} ) \implies\\
 \t1 (\exists road : Seq~~Cell @ head~~road = c_1 \land (\forall x: \nat| x \ge 1 \land x < \#road @\\
 \t1 \t1 table(\{~ road~~x , road~~x+1~\}) = \{~ path ~\} \land \\
 \t1 \t1(\exists i_1,i_2,j_1,j_2 : \nat |(i_1,j_1)=road~~x \land (i_2,j_2)=road~x+1@(i_1=i_2+1 \land j_1=j_2)\lor\\
 \t1 \t1 \t1(j_1=j_2+ \land i_1=i_2))) \land\\
 \t1 road~~\#road = c_2) 
\end{schema}

اندازه جدول توسط بازیکن قبل از شروع بازی مشخص می شود و اندازه جدول حداقل \lr{3*3} باید باشد. تعداد روح ها و دانه ها و جایگاهشان همراه با جایگاه پکمن در ابتدای بازی به شکل تصادفی انتخاب می شوند .

\begin{schema}{GameTableInit}
GameTable'\\
sampleLength? : \nat\\
sampleWidth? : \nat
\where
pacman' \notin gosts'\\
table' = sampleTable\\
pacman' = samplePacman\\
ghosts' = sampleGhosts\\
seeds' = sampleSeeds\\
length' = sampleLength?\\
width' = sampleWidth?\\ 
sampleWidth \ge 3\\
sampleLength \ge 3\\
pacman' \neq \emptyset
\end{schema}

\begin{axdef}
sampleTable : Cell \fun State
\end{axdef}

\begin{axdef}
samplePacman : Cell 
\end{axdef}

\begin{axdef}
sampleGhosts : \power~Cell 
\end{axdef}

\begin{axdef}
sampleSeeds : \power~Cell 
\end{axdef}

در این قسمت حرکت پکمن را توصیف کرده ایم.

\begin{schema}{PMoveUp}
\Delta GameTable\\
i : \nat\\
j : \nat\\
direction? : Move
\where
direction? = UP\\
table' = table\\
(i,j) = pacman\\
table (i-1,j) \neq wall
pacman' = (i-1,j)\\
ghosts' = ghosts\\
seeds' = seeds\\
length' = length\\
width' = width 
\end{schema}

\begin{schema}{PMoveDown}
\Delta GameTable\\
i : \nat\\
j : \nat
direction? : Move
\where
direction? = DOWN\\
table' = table\\
(i,j) = pacman\\
table (i+1,j) \neq wall
pacman' = (i+1,j)\\
ghosts' = ghosts\\
seeds' = seeds\\
length' = length\\
width' = width 
\end{schema}

\begin{schema}{PMoveRight}
\Delta GameTable\\
i : \nat\\
j : \nat
direction? : Move
\where
direction? = Right\\
table' = table\\
(i,j) = pacman\\
table (i,j+1) \neq wall
pacman' = (i,j+1)\\
ghosts' = ghosts\\
seeds' = seeds\\
length' = length\\
width' = width 
\end{schema}

\begin{schema}{PMoveLeft}
\Delta GameTable\\
i : \nat\\
j : \nat\\
direction? : Move
\where
direction? = LEFT\\
table' = table\\
(i,j) = pacman\\
table (i,j-1) \neq wall
pacman' = (i,j-1)\\
ghosts' = ghosts\\
seeds' = seeds\\
length' = length\\
width' = width 
\end{schema}

\begin{zed}
MovePacman == PMoveUp \lor PMoveDown \lor PMoveLeft \lor PMoveRight 
\end{zed}

در این قسمت عملات خوردن دانه توسط پکمن را توصیف کرده ایم.

\begin{schema}{EatSeed}
\Delta GameTable
\where
table' = table\\
pacman \in seeds\\
seeds' = seeds \backslash pacman\\
ghosts' = ghosts\\
length' = length\\
width' = width 
\end{schema}


در این قسمت حرکت روح ها را که قرار است تصادفی و غیر هوشمندانه باشد را توصیف کرده ایم.
\\
همان طور که در \lr{State Schema} مشخص است ، بنده در این توصیف روح ها را یک مجموعه از مختصات تعریف کردم ، به این معنا که هر کدامشان در یک خانه از جدول هستند. حال ۴ \lr{operation Schema} برای تعریف حرکت روح ها نوشتم یکی برای بالا رفتنشان ،  یکی برای پایین رفت ،  یکی برای راست رفتن و یکی برای چپ رفتن . در ادامه یکی از آن ها را توضیح میدهم.
\\
یک مجموعه \lr{temp} تعریف میکنیم از مختصات که معرف مجموعه جدیدی از مختصات روح هاست . حالا یا روح ها بالا میروند به طور همزمان و یا سر جایشان می ایستند . پس میگوییم به ازای هر مختصات از روح ها که داریم یک مختصات در مجموعه \lr{temp} داریم که یا معادل خانه بالایی آن روح است و بدین معنی است که آن روح بالا می رود و یا معادل با همان خانه ای است که روح در آن قرار دارد به این معنا که روح سرجایش میماند و تکون نمی خورد.

\begin{schema}{GMoveUp}
\Delta GameTable\\
temp : \power~Cell
\where
\#temp = \#ghosts\\
\forall c : Cell @ c \in ghosts \implies \exists t :Cell | t \in temp @ t=c \lor\\
\t1 (\exists i_1,i_2,j_1,j_2 : \nat | (i_1,j_1) = c \land (i_2,j_2) = t @ i_2 = i_1-1 \land j_1=j_2)\\
table(temp) = \{~path~\}\\
temp \in dom~~ table\\
table' = table\\
pacman' = pacman\\
ghosts' = temp\\
seeds' = seeds\\
length' = length\\
width' = width 
\end{schema}

\begin{schema}{GMoveDown}
\Delta GameTable\\
temp : \power~Cell
\where
\#temp = \#ghosts\\
\forall c : Cell @ c \in ghosts \implies \exists t :Cell | t \in temp @ t=c \lor\\
\t1 (\exists i_1,i_2,j_1,j_2 : \nat | (i_1,j_1) = c \land (i_2,j_2) = t @ i_2 = i_1+1 \land j_1=j_2)\\
table(temp) = \{~path~\}\\
temp \in dom~~ table\\
table' = table\\
pacman' = pacman\\
ghosts' = temp\\
seeds' = seeds\\
length' = length\\
width' = width 
\end{schema}

\begin{schema}{GMoveRight}
\Delta GameTable\\
temp : \power~Cell
\where
\#temp = \#ghosts\\
\forall c : Cell @ c \in ghosts \implies \exists t :Cell | t \in temp @ t=c \lor\\
\t1 (\exists i_1,i_2,j_1,j_2 : \nat | (i_1,j_1) = c \land (i_2,j_2) = t @ i_2 = i_1 \land j_1 + 1=j_2)\\
table(temp) = \{~path~\}\\
temp \in dom~~ table\\
table' = table\\
pacman' = pacman\\
ghosts' = temp\\
seeds' = seeds\\
length' = length\\
width' = width 
\end{schema}

\begin{schema}{GMoveLeft}
\Delta GameTable\\
temp : \power~Cell
\where
\#temp = \#ghosts\\
\forall c : Cell @ c \in ghosts \implies \exists t :Cell | t \in temp @ t=c \lor\\
\t1 (\exists i_1,i_2,j_1,j_2 : \nat | (i_1,j_1) = c \land (i_2,j_2) = t @ i_2 = i_1 \land j_1 - 1=j_2)\\
table(temp) = \{~path~\}\\
temp \in dom~~ table\\
table' = table\\
pacman' = pacman\\
ghosts' = temp\\
seeds' = seeds\\
length' = length\\
width' = width 
\end{schema}

حال در این بخش سعی کردم به نحوی حرکت تصادفی روح ها را بیان کنم 

\begin{zed}
MoveGhosts == \exists m:Move @\\
\t1 (m = UP\land GMoveUp)\lor (m = DOWN\land GMoveDown)\lor\\
\t1 \t1 (m = RIGHT\land GMoveLeft)\lor (m = LEFT\land GMoveRight)
\end{zed}

در این بخش هم حالت بردن و باختن بازی را توصیف کرده ایم

\begin{schema}{Win}
\Xi GameTable\\
output! : Message
\where
seeds = \emptyset\\
output! = win
\end{schema}

\begin{schema}{Defeated}
\Xi GameTable\\
output! : Message
\where
pacman \in ghosts\\
output! = defeated
\end{schema}

\begin{zed}
End == Win \lor Defeated
\end{zed}

\begin{zed}
PacmanGame == MovePacman \lor EatSeed \lor MoveGhosts \lor End
\end{zed}


\end{document}

